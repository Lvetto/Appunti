\documentclass{article}

% importa moduli
\usepackage{graphicx} % Required for inserting images
\usepackage{dsfont}
\usepackage{amssymb}
%\usepackage{cases}
\usepackage{ifthen}
\usepackage{geometry}
 \geometry{
 a4paper,
 total={180mm,280mm},
 left=15mm,
 top=5mm,
 }
\usepackage{stackengine}
\usepackage{amsmath} 
\usepackage{mathtools}
\newcommand\ubar[1]{\stackunder[1.2pt]{$#1$}{\rule{.8ex}{.075ex}}}
\usepackage{graphicx}
\usepackage{wrapfig}


\title{Termodinamica}
\author{Luca Vettore}
\date{November 2023}

\begin{document}

% importa moduli
\usepackage{ dsfont }
\usepackage{ amssymb }
\usepackage{cases}
\usepackage{ifthen}
\usepackage{geometry}
 \geometry{
 a4paper,
 total={180mm,280mm},
 left=15mm,
 top=5mm,
 }
\usepackage{stackengine}
\usepackage{amsmath} 
\usepackage{mathtools}
\newcommand\ubar[1]{\stackunder[1.2pt]{$#1$}{\rule{.8ex}{.075ex}}}
\usepackage{graphicx}
\usepackage{wrapfig}
%\graphicspath{ {./ims/} }

% importa immagini
\newcommand{\im}[2]{
\begin{center}
\includegraphics[width=#1\textwidth]{#2}
\end{center}
}

\newcommand{\ims}[3]{
\begin{center}
\includegraphics[width=#1\textwidth]{#2}
\includegraphics[width=#1\textwidth]{#3}
\end{center}
}

\newcommand{\imsd}[4]{
\begin{center}
\includegraphics[width=#1\textwidth]{#3}
\includegraphics[width=#2\textwidth]{#4}
\end{center}
}

% insiemi numerici
\newcommand{\R}{\mathds{R}}
\newcommand{\N}{\mathds{N}}
\newcommand{\C}{\mathds{C}}

% derivate parziali 
\newcommand{\pd}[2]{\frac{\partial #1}{\partial #2}}

% indice
\newcommand{\Index}{
\newpage
\renewcommand*\contentsname{Indice}
\tableofcontents
}

% teoremi
\newcommand{\teo}[3]{
\textbf{Teorema #1}\\
#2\\
\ifthenelse{\equal{#3}{}}{}{\textbf{Dimostrazione}\\
#3}
}

% matrici
\newcommand{\mat}[1]{{
  \tiny\arraycolsep=0.3\arraycolsep\ensuremath{\begin{pmatrix}#1\end{pmatrix}}}}
  
% nabla
\newcommand{\Nabla}[]{\ubar{\nabla}}

% cambia altezza righe tabelle
\renewcommand{\arraystretch}{1.5}



\maketitle

\section{Introduzione}
Lo scopo di questo documento è quello di fornire un percorso alternativo per lo studio della termodinamica. Le varie sezioni affrontano in modo approfondito un macro-argomento, combinando informazioni prese da libri e internet, rielaborate e presentate in un modo ritenuto più logico dall'autore. Questo documento è inteso come uno strumento aggiuntivo per lo studio della materia, ma richiede già una conoscenza pregressa degli argomenti trattati. L'ordine delle sezioni è abbastanza casuale ed è possibile vi siano ripetizioni o vengano usati concetti definiti in una sezione successiva.





\section{La temperatura}
\subsection{Definizione e proprietà}
Un sistema è detto chiuso quando è in grado di scambiare energia, ma non materia con l'ambiente esterno. Mettendo a contatto due sistemi chiusi, questi potranno solamente scambiare energia sotto forma di lavoro o calore.\\
Consideriamo due sistemi chiusi che non possono scambiare lavoro (ad esempio contenitori diatermani con pareti rigide). Mettendo a contatto i due sistemi, si osservano comunque trasformazioni fisiche (cambiamenti nelle grandezze che li descrivono, ad es volume o pressione). Se non avvengono trasformazioni, allora i due sistemi sono detti in equilibrio termico.\\\\
Introduciamo ora il principio 0 della termodinamica:
\begin{center}
    \textit{Dati tre sistemi A,B,C, se A è in equilibrio termico con B e B è in equilibrio termico con C, allora A è in equilibrio termico con C.}
\end{center}
Deve quindi esistere una proprietà comune tra i 3 sistemi che ne descrive l'equilibrio termico. Questa proprietà è la temperatura ed è una grandezza intensiva (non è additiva rispetto ai sistemi).\\\\
Come possiamo quantificare la temperatura? Prendiamo due sistemi A e B, descritti da due coordinate termodinamiche $(X_i^j,Y_i^j)$. Per prima cosa prepariamo il sistema A negli stati $A_j$ e li mettiamo a contatto con B fino al raggiungimento dell'equilibrio termico. Possiamo quindi misurare le coordinate $(X_a^j,Y_a^j)$ e rappresentarle su un piano cartesiano. Ripetiamo il procedimento con gli stati $B_j$ mettendoli a contatto con A. Otterremo due curve, dette isoterme, dove ogni stato è in equilibrio termico con tutti gli altri ($\Leftarrow$ principio 0).\\
Con un procedimento analogo possiamo tracciare alcune isoterme per uno dei due sistemi. Fissiamo $Y=y_0$, i punti corrispondenti sulle varie isoterme permettono di associare $X^j$ a varie temperature $t^j$. Scegliamo due di queste isoterme come riferimento, troviamo $X_1,X_2$ intersezioni delle isoterme con $Y=y_0$ e dividiamo questo intervallo in un certo numero di gradi. La temperatura sarà quindi una funzione $t=\tau(x)$, di solito scelta lineare per comodità.\\
La temperatura così definita è quindi indice dello stato termico di un sistema.

\subsection{Esempi di scale termometriche}
\subsubsection{Termometri analogici e Celsius ($^oC$)}
Nella pratica risulta comodo prendere come riferimento due sistemi in cui vi è equilibrio tra varie fasi (ad es solido e liquido). Durante una transizione di fase, un sistema tende a mantenere la stessa temperatura e il calore che gli viene fornito è utilizzato per rompere i legami chimici tra la molecole (calore latente).\\\\
La scala Celsius prende come riferimento il punto di fusione dell'acqua e il suo punto di ebollizione. Una scelta pratica per la grandezza termometrica da utilizzare può essere fatta prendendo in considerazione la dilatazione termica. Un corpo con una dimensione molto maggiore delle altre, tenderà ad espandersi all'aumentare della temperatura secondo la legge:
$$ l(t)\simeq l_0(1+\alpha\Delta t) $$
Misurando l'oggetto in equilibrio termico con i due punti di riferimento, è possibile ottenere una scala di temperature in cui a una temperatura in gradi $^oC$ corrisponde ad una certa lunghezza. Questo è il principio di funzionamento dei termometri analogici.

\subsubsection{Termometri a gas e Kelvin ($K$)}
Possiamo definire una scala termometrica assoluta usando un gas e la legge di Gay-Lussac. Un gas mantenuto a volume costante segue la legge:
$$ \frac{P_0}{T_0}=\frac{P_1}{P_0} \Rightarrow P(T) \propto T $$
Possiamo usare lo stesso riferimento di prima e ottenere una scala termometrica assoluta (perchè legata al comportamento dei gas). Dividendo nuovamente l'intervallo in 100 gradi e aggiungendo la condizione $0^oC=273,16K$ otteniamo una facile corrispondenza tra Celsius e Kelvin.

\subsubsection{Macchine termiche e scala termodinamica}
Il primo teorema di Carnot prevede che le quantità di calore scambiate durante un ciclo di Carnot dipenda esclusivamente dalle temperature delle due isoterme. Possiamo definire un'ulteriore scala termometrica, nota come scala termodinamica, partendo da questo risultato.\\
Consideriamo tre macchine termiche $M_i$. La prima opera tra le temperature $t_1,t_2$, quindi applicando il teorema di Carnot:
$$ \left| \frac{Q_1}{Q_2} \right| = f(t1, t2) $$
La seconda opera tra $t_1, t_3$, da cui:
$$ \left| \frac{Q_1}{Q_3} \right| = f(t1, t3) $$
e la terza tra $t_2,t_3$:
$$ \left| \frac{Q_2}{Q_3} \right| = f(t2, t3) $$
con dei semplici passaggi algebrici:
$$ \left| \frac{Q_1}{Q_2} \right| = \left| \frac{Q_1}{Q_3} \right| \cdot \left| \frac{Q_3}{Q_2} \right| = \frac{f(t_1, t_3)}{f(t_2,t_3)} = \frac{\theta(t_1)}{\theta(t_2)} = \frac{T_1}{T_2} \text{\;(dal T di Carnot)} $$
quindi:
$$ T=T_0\left| \frac{Q}{Q_0} \right| $$
con un'appropriata scelta di parametri, questa scala può essere fatta coincidere con quella assoluta.




\newpage
\section{I gas}

\subsection{I gas perfetti}
Il primo modello di gas che studiamo è quello di gas ideale o perfetto. Questo modello si basa su alcune ipotesi:
\begin{itemize}
    \item il gas è formato da molecole puntiformi non interagenti
    \item gli urti tra le particelle sono elastici e tra due urti una particella si muove di moto rettilineo uniforme
    \item la distribuzione delle particelle è isotropa
    \item non vi sono attriti interni o con le pareti del contenitore. Equivalentemente, l'energia si conserva
\end{itemize}
Un gas reale è ben approssimato da questo modello a basse pressioni e alte temperature.\\
Un gas perfetto è descritto dalla terna di coordinate termodinamiche $(P,V,T)$.

\subsubsection{Osservazioni sperimentali ed equazione di stato}
Partiamo da una serie di osservazioni e leggi sperimentali, da cui poi ricaveremo un risultato più generale.
\begin{itemize}
    \item \textit{Legge di Boyle-Mariotte}: durante una trasformazione a temperatura costante vale: $$ PV = const \quad\text{(t costante)} $$
    \item \textit{Legge di Charles}: in una trasformazione a pressione costante vale:
    $$ V=V_0(1+\alpha\Delta t) \quad\text{(P costante)} $$
    \item \textit{Legge di Gay-Lussac}: durante una trasformazione a volume costante vale:
    $$ P=P_0(1+\alpha\Delta t)\quad \text{V costante} $$    
\end{itemize}
Dove $\alpha$ è una costante universale (indipendente dal gas) e vale $\alpha = \frac{1}{273,16}$.\\\\
Introducendo la scala termometrica assoluta e la conversione $T=t\frac{K}{^oC}+T_0$ con $T_0=273,16K$, possiamo riscrivere le ultime due leggi:
\begin{itemize}
    \item \textit{Legge di Charles}:
    $$ V=V_0(1+\frac{T-T_0}{T_0})=V_0\frac{T}{T_0} \Rightarrow \frac{V}{T}=\frac{V_0}{T_0}\quad\text{P costante}$$

    \item \textit{Legge di Gay-Lussac}:
    $$ P=P_0(1+\frac{T-T_0}{T_0})=P_0\frac{T}{T_0}\Rightarrow\frac{P}{T}=\frac{P_0}{T_0}\quad\text{V costante} $$
\end{itemize}
Combinando le leggi di Boyle-Mariotte, Charles e Gay-Lussac, possiamo derivare l'equazione di stato dei gas perfetti. Iniziamo con una trasformazione isobara (a pressione costante) e successivamente applichiamo una trasformazione isoterma (a temperatura costante). Durante la trasformazione isobara, secondo la legge di Charles, il volume di un gas è direttamente proporzionale alla sua temperatura assoluta, ovvero $ V \propto T $. Successivamente, applicando la trasformazione isoterma, secondo la legge di Boyle-Mariotte, il prodotto di pressione e volume è costante, ovvero $ PV = \text{costante} $.\\\\
Combinando queste relazioni, otteniamo che $ PV \propto T $. Ulteriori osservazioni sperimentali ci portano a esprimere tale risultato come:
$$PV=nRT$$
dove $n$ rappresenta il numero di moli e $R$ è la costante dei gas perfetti. Questa relazione è nota come legge di stato dei gas perfetti.\\\\
La costante dei gas perfetti $ R $ può essere espressa in diverse unità. In termini di energia, quando si utilizza il Joule come unità di energia, il valore di $ R $ è approssimativamente $ 8.314 \, \text{J/(mol·K)} $. Se invece si utilizza la Caloria il valore di $ R $ è circa $ 1.987 \, \text{cal/(mol·K)} $. Questi valori riflettono le diverse scale energetiche e sono utili a seconda del contesto dell'analisi termodinamica.

\subsubsection{Teoria cinetica}
Possiamo studiare il comportamento di un gas ideale analiticamente, partendo dalla sua struttura microscopica e ricavandone importanti relazioni sulle grandezze macroscopiche.\\\\
Consideriamo la singola particella del gas in esame, essa sarà caratterizzata da una velocità $v_i=\begin{bsmallmatrix} v_i^x\\v_i^y\\v_i^z  \end{bsmallmatrix}$e avrà una quantità di moto $q_i=m\cdot v_i$.\\
Durante un urto con la parete del contenitore (per semplicità consideriamo un contenitore cubico di lato $l$, ma un analisi più dettagliata permette di estendere questi risultati a contenitori di forma arbitraria), la particella scambia una quantità di moto pari a $\Delta q_i^j=2m\cdot v_i^j$.\\
Il tempo tra due urti successivi con le pareti (ignorando gli urti interni, ma anche qua un'analisi più dettagliata permette di generalizzare il risultato) è dato da $\Delta\tau_i^j=\frac{2l}{v_i^j}$.\\
Combinando i due risultati precedenti possiamo calcolare la forza media e di conseguenza la pressione applicata sulla parete $j$ come:
$$ P_i^j=\frac{F_i^j}{l^2}=\frac{m(v_i^j)^2}{l^3} $$
Introducendo la densità del gas: $ \rho=\frac{Nm}{l^3} $ (dove $N$ è il numero di particelle)\\
e la velocità quadratica media: $<v_j^2>=\frac{\sum_i(v_i^j)^2}{N}$
possiamo riscrivere la pressione come:
$$ P^j=\rho<v_j^2>$$
Imponendo l'isotropia del gas e definendo $v_{qm}=\sqrt{\sum_j<v_j^2>}$ si ottiene:
$$ P_j=P_k\Rightarrow<v_j^2>=\frac{1}{3}v^2_{qm} $$
$$ \Rightarrow P=P_j=\frac{1}{3}\rho v^2_{qm} $$
Assumendo che le particelle siano identiche e che l'energia delle particelle sia esclusivamente cinetica, l'energia interna del gas vale:
$$ U = \frac{Nm}{2}\frac{\sum_i v^2_{qm}}{N} $$
Usando la relazione appena trovata per la pressione, si può ricavare $v^2_{qm}=\frac{3P}{\rho}$, imponendo la legge di stato del gas ideale $PV=nRT$ è possibile riscrivere l'energia interna come:
$$ U=\frac{3}{2}nRT $$
Dall'equazione di stato e dalla definizione trovata per la pressione, si può ricavare la definizione cinetica di temperatura:
$$ T = \frac{\mathcal{M} v^2_{qm}}{3R} $$
dove $\mathcal{M}$ è la massa molare del gas.

\subsubsection{Distribuzione delle velocità}
Studiamo ora la distribuzione delle velocità all'interno del gas. Questo studio ci permetterà poi di dimostrare il teorema dell'equipartizione dell'energia ed estendere i risultati della sezione precedente a gas pluriatomici.\\\\
Consideriamo un contenitore cilindrico di gas (come di consueto, la scelta è fatta per semplificare i conti, ma i risultati hanno validità generale). Per un volume infinitesimo di gas compreso tra la quota $z$ e $z+dz$ la condizione di equilibrio meccanico risulta:
$$ PA=(p+dp)A+dm\;g = (P+dp)\rho Adzg $$
$$ \Rightarrow Pa-Pa = Adp+\rho Adzg$$
$$ \Rightarrow dp = -\rho dzg $$
La densità vale $ \rho=\frac{n\mathcal{M}}{nRT} $, quindi:
$$ \frac{dp}{P} = -\frac{\mathcal{M}}{RT}gdz $$
$$\Rightarrow \int_{P_0}^P\frac{dp}{P}=-\int_0^z\frac{\mathcal{M}}{RT}gdz  $$
$$\Rightarrow \log\left(\frac{P}{P_0}\right)=-\frac{\mathcal{M}gz}{RT} $$
da cui si ricava l'equazione barometrica:
$$P(z)=P_0\exp\left(-\frac{\mathcal{M}gz}{RT}\right)=P_0\exp\left(-\frac{mgdz}{k_bT}\right)$$
Abbiamo già dimostrato che vale $\rho\propto P$, quindi:
$$ \rho(z)=\rho_0\exp\left(-\frac{mgdz}{k_bT}\right)$$
Le particelle alla quota $z$ saranno quelle che inizialmente avevano energia cinetica pari al potenziale gravitazionale a tale quota: $T=\frac{m}{2}v_z^2=mgz$. Sostituendo questa relazione nell'equazione barometrica otteniamo:
$$ \rho(v_z)=\rho_0\exp\left(-\frac{mv_z^2}{2k_bT}\right) $$
Definiamo una funzione $f$ che rappresenta la probabilità di una particella di trovarsi alla quota $z$: $$f(z)dz=\frac{dN(z)}{N_{tot}}\propto\rho(z)dz$$
$$\Rightarrow f(v_z)dv_z\propto\rho(v_z)dv_z $$
$$\Rightarrow f(v_z)=f_0\exp\left(-\frac{mv_z^2}{2k_bT}\right) $$
per un certo $f_0$.\\
Il valore di $f_0$ può essere trovato imponendo la condizione di normalizzazione della distribuzione di probabilità:
$$ \int_\R f(v_z)dv_z=1\Rightarrow f_0=\sqrt{\frac{m}{2\pi k_bT}} $$
Le distribuzioni di probabilità per le componenti della velocità devono essere indipendenti. Inoltre, il gas è isotropo per ipotesi. Combinando queste due condizioni, possiamo trovare la distribuzione della velocità (come vettore). Osserviamo che questa deve dipendere solamente dal modulo, quindi possiamo scrivere:
$$ f(v)=\left(\frac{m}{2\pi k_bT}\right)^{\frac{3}{2}}4\pi v^2\exp\left(-\frac{mv^2}{2k_bT}\right) $$
che è una gaussiana centrata in $v_p=\sqrt{\frac{2RT}{\mathcal{M}}}$.\\\\
Un procedimento simile può essere seguito per ricavare la distribuzione delle energie cinetiche, da cui segue che:
$$ \bar{E_i}=\frac{1}{2}k_bT $$
Questo ci permette di generalizzare il risultato trovato per l'energia interna di un gas monoatomico, a un gas formato da particelle con $f$ gradi di libertà:
$$ U = \frac{f}{2}nRT $$
Per un gas monoatomico $f=3$, per un gas biatomico $f=5$, per gas formati da più di due atomi $f=6$.


\subsection{Equilibrio}
Un sistema è detto in equilibrio termodinamico se rispetta 3 condizioni:
\begin{itemize}
    \item \textit{Equilibrio meccanico:} tutte le forze che agiscono sul sistema si equilibrano
    \item \textit{Equilibrio termico:} la temperatura del sistema è uniforme
    \item \textit{Equilibrio chimico:} la composizione del sistema rimane costante
\end{itemize}
Un sistema in equilibrio termodinamico può essere descritto da una funzione di stato:
$$ f(X_1,...,X_n) = 0 $$
nel caso di un gas perfetto questa è data dall'equazione di Clapeyron in forma implicita:
$$ PV-nRT=0 $$


\subsection{Trasformazioni}
Quando un sistema passa dallo stato $A$ allo stato $B$ si dice che subisce una trasformazione, indicata come $A\rightarrow B$. Questa trasformazione può essere classificata in diversi modi in base alle sue caratteristiche:
\begin{itemize}
    \item \textit{Trasformazione quasi-statica:} una trasformazione in cui il sistema passa esclusivamente per stati di equilibrio. Se la trasformazione viene fermata prima di essere completata, il sistema rimane nello stato che ha raggiunto.
    \item \textit{Trasformazione reversibile:} una trasformazione quasi-statica la cui inversa è quasi-statica. Questa trasformazione non provoca modificazioni nell'ambiente esterno al sistema.
    \item \textit{Trasformazione spontanea:} una trasformazione che avviene su un sistema isolato in assenza di interventi esterni. Un esempio è il passaggio di calore tra un corpo più caldo e uno più freddo.
    \item \textit{Trasformazione irreversibile:} una trasformazione che porta un sistema tra due stati di equilibrio, ma non passa per ulteriori stati di equilibrio.
    \item \textit{Trasformazione lontana dall'equilibrio:} una trasformazione che avviene in un sistema non isolato e non chiuso.
\end{itemize}



\subsection{Lavoro e calore}

\subsubsection{Trasformazioni di un gas}
Un sistema termodinamico può scambiare energia sotto forma di lavoro. In questi appunti viene usata la seguente convenzione sul segno:
\begin{itemize}
    \item $L>0$: il sistema compie lavoro
    \item $L<0$: viene compiuto lavoro sul sistema
\end{itemize}
Un gas (ideale) può compiere o subire lavoro attraverso ad espansioni o compressioni quasi-statiche. In questo caso la forza a compiere lavoro è quella dovuta alla pressione e vale:
$$ \delta L=pdV $$
Il differenziale di lavoro non è necessariamente esatto, poichè la quantità di lavoro compiuto dipende dalla trasformazione eseguita.\\\\
Un gas può compiere lavoro anche subendo trasformazioni irreversibili, in tal caso però non è possibile definire lo stato del gas durante la trasformazione (non essendo stati di equilibrio non è detto che le grandezze che lo caratterizzano siano uniformi). In questo caso è possibile usare il lavoro compiuto dall'ambiente esterno per calcolare quello compiuto dal gas. La relazione che si ottiene in questo caso è identica a quella ottenuta per trasformazioni reversibili (sostituendo la pressione del gas con quella dell'ambiente esterno), ma è importante notare che il procedimento seguito è differente.\\\\
La quantità di lavoro totale compiuta dal gas passando dallo stato $A$ allo stato $B$ è data dall'integrale del differenziale appena ottenuto:
$$ L_{AB} = \int_{A}^{B}\delta L=\int_{A}^{B}pdV$$
e possiamo notare che non dipende esclusivamente dagli stati iniziale e finale, ma è necessario considerare la trasformazione eseguita.\\\\
Il lavoro compiuto in una trasformazione reversibile risulta sempre maggiore o uguale a quello compiuto in una trasformazione irreversibile. Infatti, per eseguire una trasformazione reversibile è necessario far variare pressione e volume con continuità tra i valori nei due stati, mentre per una trasformazione irreversibile il lavoro è calcolato a partire dalla pressione esterna che è di norma costante. Questo risultato verrà poi formalizzato introducendo il secondo principio della termodinamica e il concetto di entropia.

\subsubsection{Primo principio e relazione lavoro-calore}
Consideriamo ora un sistema termodinamico chiuso su cui viene compiuto del lavoro. Dal teroema dell'energia cinetica:
$$ \Delta T = L_{int} + L_{est} $$
dove $T$ è lenergia cinetica, mentre $L_{int}$ e $L_{est}$ sono rispettivamente i lavori dovuti a forze interne ed esterne.\\\\
Le forze interne sono solitamente di tipo conservativo (forze elettromagnetiche e di coesione), possiamo quindi associarvi un potenziale che risulta essere una funzione di stato:
$L_{int}=-\Delta \mathcal{U} $
Se il sistema si trova in quiete macrofisica, possiamo definire la sua energia cinetica come somma delle energie cinetiche delle sue componenti e scrivere la sua energia interna come:
$$ U=\mathcal{U} +\mathcal{T}  $$
$U$ risulta quindi essere una funzione di stato (dipende solo dallo stato del sistema e non da come questo è stato raggiunto). Possiamo quindi scrivere la relazione:
$$ L_{est} = \Delta U $$
Se anche le forze esterne al sistema sono di natura conservativa, possiamo definire un altro potenziale $W$ e riformulare la relazione precedente come:
$$ L_{est}=-\Delta W\Rightarrow \Delta(U+W)=0 $$
Introduciamo ora forze esterne di tipo non conservativo, come ad esempio attriti. Consideriamo un esperimento noto come mulinello di Joule. Un contenitore con un rivestimento adiatermano (calorimetro) è riempito di acqua in equilibrio termico con l'ambiente esterno. Nel volume d'acqua è immerso un mulinello a palette (un dispositivo in grado di ruotare e trasferire energia al sistema attraverso ad attriti). Il mulinello è collegato a una massa in moto per effetto della gravità. Una volta fermata la massa, si osserva un aumento di temperatura del sistema. Rimuovendo il rivestimento adiatermano e lasciando raggiungere l'equilibrio termico al sistema, si osserva una diminuzione della temperatura fino al valore iniziale. Essendo l'energia interna una funzione di stato deve valere:
$$ \Delta U = \Delta U_1+\Delta U_2 = 0\Rightarrow\Delta U_2<0 $$
dove $\Delta U_1$ indica la variazione di energia interna nella prima parte dell'esperimento (massa in moto) e $\Delta U_2$ nella seconda (raggiungimento equilibrio). Se provassimo ad applicare la relazione ottenuta per forze conservative otterremmo:
$$ L_{est} = L_{est}^1+L_{est}^2\neq0\Rightarrow\Delta W\neq0\Rightarrow\Delta(U+W)\neq0 $$
Questo è dovuto al fatto che nella seconda parte dell'esperimento è avvenuto uno scambio di energia senza che venisse compiuto lavoro. Per ripristinare il bilancio energetico, è necessario considerare il calore come una qualche forma di lavoro. Oltre al lavoro coordinato compiuto dalla massa, possiamo introdurre un lavoro caotico legato alla diminuzione di energia cinetica delle particelle del sistema durante il raggiungimento dell'equilibrio termico. In questo modo possiamo riscrivere:
$$ L_{est} = L_{co}+L_{ca} $$
e dovrà esistere una relazione del tipo
$$ L_{ca} = f(Q) $$
che Joule dimostrò sperimentalmente essere di tipo lineare: $L_ca\propto Q$. Esprimendo il calore in joule, la costante di proporzionalità sparisce (vale 1). Possiamo quindi scrivere:
$$ L_{co}+L_{ca}=\Delta U $$
$$\Rightarrow L_{co}+Q=\Delta U \quad\text{(Q espresso in joule)} $$
Considerando stati di equilibrio termodinamico, il sistema deve essere anche in equilibrio meccanico, quindi $L=-L_{co}$ e possiamo enunciare il primo principio della termodinamica:
\begin{center}
    \textit{Per un sistema in quiete macrofisica, la quantità di calore scambiata con l'ambiente esterno è pari alla somma del lavoro compiuto e della variazione della sua energia interna}
\end{center}
Che può essere espresso in formule come:
$$ Q=L+\Delta U $$
$$ \delta Q=\delta L + dU $$
Questo risultato ha validità generale, ma è particolarmente utile nello studio dei gas e delle loro trasformazioni. Per un gas ideale normalmente conosciamo l'espressione analitica della sua energia interna e per trasformazioni quasi statiche anche il differenziale di lavoro, possiamo quindi usare il primo principio per legare la quantità di calore che scambia con la variazione delle grandezze che lo descrivono.


\subsubsection{Calori specifichi}
Il calore specifico è definito come la quantità di calore che occore fornire a un sistema per aumentarne la temperatura di un grado. Nel caso di un gas (ideale), questa quantità dipende dalla particolare trasformazione seguita. Possiamo però definire dei calori specifichi medi validi per alcune trasformazioni (che mantengono costante una delle grandezze).\\
Per un gas risulta più comodo definire i calori specifici molari, cioè la quantità di calore da fornire a una mole di gas oer aumentarne la temperatura di un grado. Calcolare il calore specifico (per unità di massa) partendo da quello molare è poi molto semplice, è sufficiente dividerlo per la massa di una mole di gas.


\newpage
\section{Macchine termiche}
\subsection{Introduzione e definizioni}
Nella sezione precedente abbiamo visto che esiste una relazione che lega calore e lavoro. Entrambi sono forme di trasferimento di energia, deve quindi essere possibile convertire uno nell'altro.\\\\
L'approccio più semplice consisterebbe nello sfruttare un'espansione isoterma ($\Delta U=0$) per convertire tutto il calore fornito in lavoro, ma così facendo per continuare a estrarre lavoro dovremmo aumentare sempre di più il volume del gas e risulterebbe scomodo da mettere in pratica.\\\\
Le macchine termiche sfruttano un ciclo termodinamico in cui il gas viene riportato allo stato iniziale dopo aver convertito calore in lavoro.
\im{0.5}{"ims/ciclo"}
Consideriamo un ciclo generico, in cui il gas passa ciclicamente tra 3 stati $A,B,C$. Scegliamo per semplicità gli stati $B,C$ in modo da essere rispettivamente quelli in cui il gas raggiunge la temperatura più alta e più bassa. Durante la trasformazione $A\rightarrow B$, la temperatura aumenta, supponiamo che aumentino anche pressione e volume (scelta arbitraria, il ragionamento funzione anche scambiando le caratteristiche dei 3 stati). Per questa trasformazione avremo $\Delta U_1>0$ e $L_1>0$, quindi il gas dovra assorbire un calore $Q_1>0$ (dal primo principio $Q=\Delta U + L$). Lo stato $B$ avrà temperatura, volume e pressione minori, quindi $\Delta U_2<0$ e $L<0$, scambierà quindi un calore $Q_2<0$. Per riportare il gas allo stato originale, il gas dovrà scambiare $L_3<0$ e aumentare la sua energia interna (la temperatura è magiore di quella minima), quindi $\Delta U >0$. Non possiamo quindi dire nulla a priori sul segno del calore. Se ora applichiamo il primo principio all'intero ciclo:
$$ L = Q - \Delta U = Q = Q_{assorbito} - |Q_{ceduto}| $$
Assumendo che il gas abbia compiuto un lavoro totale positivo:
$$ L = Q_{assorbito} - |Q_{ceduto}| > 0\Rightarrow L < Q_{abs}$$
Vedremo in seguito come questo fatto abbia validità generale. Non è possibile convertire totalmente il calore in lavoro in maniera ciclica.\\\\
Il rendimento di una macchina termica è definito come rapporto tra il lavoro compiuto e il calore assorbito:
$$ \eta=\frac{L}{Q_{abs}} $$
Una macchina frigorifera usa un ciclo termodinamico per trasferire calore da una sorgente calda a una fredda, usando un lavoro esterno. Il coefficiente di prestazione di una macchina frigorifera è definito come:
$$ \eta_f = \frac{Q_f}{|L|} $$

\subsection{Primo teorema di Carnot}
\im{0.5}{ims/Carnot}
Il ciclo di Carnot è composto da due trasformazioni isoterme e due adiabatiche. Chiamando $Q_c$ il calore scambiato con la sorgente calda (l'isoterma a temperatura maggiore) e $Q_f$ quello scambiato con la sorgente fredda (l'isoterma a temperatura minore), dal primo principio otteniamo:
$$ L = Q_c - |Q_f| $$
$$ \Rightarrow \eta = 1 - \frac{|Q_f|}{Q_c} $$
Il ciclo scambia calore solo durante le due isoterme e in particolare vale:
$$ Q_c=L_{ab} \qquad Q_f=L_{CD}$$
Risolvendo gli integrali per i lavori delle due isoterme e ricavando le grandezze negli stati applicando l'equazione di Poisson alle adiabatiche (molto più semplice di quello che sembra), otteniamo:
$$ \frac{|Q_f|}{Q_c} = \frac{T_f}{T_c} $$
e quindi il rendimento vale:
$$ \eta = 1 - \frac{T_f}{T_c} $$
che è l'enunciato del primo teorema di Carnot.\\\\
Le trasformazioni usate da un ciclo di Carnot sono reversibili, è quindi possibile invertire il ciclo per ottenere una macchina frigorifera con coefficiente di prestazione:
$$ \eta_f = \frac{T_c}{T_c-T_f} $$
Il ciclo di Carnot frigorifero può anche essere usato come pompa di calore, per fornire calore a una sorgente calda estraendolo da una fredda, usando lavoro. Questo ciclo è identico a quello frigorifero, ma cambia lo scopo per cui viene usato. Il coefficiente di prestazione sarà quindi definito come:
$$ \eta_{pc} = \frac{|Q_c|}{|L|} $$
e vale:
$$ \eta_{pc} = \frac{T_c}{T_c-T_f} $$

\subsection{Secondo teorema di Carnot}
Nell'introduzione di questa sezione abbiamo parlato di come sia impossibile trasformare tutto il calore in lavoro senza modificare il sistema o l'ambiente. Questa affermazione può essere formalizzata attraverso al secondo principio della termodinamica. Questo principio può essere formulato in modi diversi. Di seguito sono riportati gli enunciati di Clausius e di Kelvin-Planck.
\begin{center}
    Clausius: \textit{Non è possibile realizzare un processo il cui unico risultato sia il trasferimento di calore da un corpo freddo ad uno più caldo}
\end{center}

\begin{center}
    Kelvin-Planck: \textit{Non è possibile realizzare un processo il cui unico risultato sia quello di assorbire calore da un'unica sorgente e trasformarlo integralmente in lavoro}
\end{center}
Supponiamo esista una macchina termica $G$ con rendimento maggiore a quello di un ciclo di Carnot $C$ tra le stesse sorgenti. La macchina $G$ è quindi in grado di compiere la stessa quantità di lavoro assorbendo meno calore:
$$ \eta_G=\frac{L}{Q_1'}>\eta_C=\frac{L}{Q_1} \Rightarrow Q_1'<Q_1$$
Supponiamo ora di invertire $C$ per ottenere una macchina frigorifera che richiede lavoro $L$ per assorbire calore $|Q_2|$ dalla sorgente fredda e fornire calore $-Q_1$ alla sorgente calda. Combiniamo le due macchine in modo da alimentare $C^{-1}$ con il lavoro prodotto da $G$. La macchina che otteniamo non richiede lavoro esterno (i due lavori sono uguali per costruzione) e scambia con le due sorgenti $T_1$ (calda) e $T_2$ (fredda) i calori:
$$ Q_1 - Q_1' < 0 \quad\text{con $T_1$} $$
$$ |Q_2|-|Q_2'| > 0 \quad\text{con $T_2$} $$
Senza richiedere lavoro estrae quindi calore dalla sorgente fredda per fornirlo a quella calda, violando l'enunciato di Clausius del secondo principio. Deve quindi necessariamente essere:
$$  \eta_G\leq\eta_C $$
per qualsiasi macchina termica $G$. Un analogo ragionamento ci permette di mostrare che l'uguaglianza vale per qualsiasi $G$ reversibile.

\subsection{Entropia?}




\section{I princìpi della termodinamica}

\subsection{Principio zero}
\begin{center}
    \textit{Se un sistema A è in equilibrio termico con un sistema B, e B è in equilibrio termico con un sistema C, allora A è in equilibrio termico con C}
\end{center}
Questo principio permette di definire il concetto di temperatura, come indicatore universale di equilibrio termico.

\subsection{Primo principio}
\begin{center}
    \textit{Per un sistema termoodinamico in quiete macrofisica la quantità di calore scambiata con l'esterno è pari alla somma del lavoro compiuto e della variazione di energia interna}
\end{center}
Che può essere espresso in formule come:
$$ Q=L+\Delta U $$
o in forma differenziale:
$$ \delta Q=\delta L + dU $$
Questo principio implica che l'energia di un sistema debba essere conservata. In effetti può essere ricavato partendo dal teorema dell'energia cinetica (altra legge di conservazione, vedi sezione dedicata ai gas) e da alcune osservazioni sperimentali.\\\\
Un altro modo per arrivare a questo principio è partire osservando sperimentalmente che nonostante il lavoro e il calore scambiati nel passaggio da uno stato $A$ a uno stato $B$ dipendano dalla trasformazione eseguita, la quantità $(Q-L)_{A\rightarrow B}$ dipende solo dai due stati. Possiamo quindi scrivere:
$$ (Q-L)_{AB} = f(A, B) $$
Considerando una trasformazione reversibile possiamo ottenere che:
$$ f(A,B)=-f(B,A) $$
Prendendo un terzo stato $O$, semplici passaggi algebrici mostrano che:
$$ f(A,B)=f(O,B)-f(O, A) $$
Fissando lo stato di riferimento $O$ e definendo $f(O,X)=U(X)$ otteniamo:
$$ f(A,B)=(Q-L)_{AB}=U(B)-U(A)=\Delta U_{AB} $$
dove $U(X)$ è definita a meno di una costante arbitraria $U(O)$.

















\Index

\end{document}
