\documentclass{article}
\usepackage[utf8]{inputenc}
\usepackage{ dsfont }
\usepackage{ amssymb }
\usepackage{cases}
\usepackage{geometry}
 \geometry{
 a4paper,
 total={190mm,257mm},
 left=10mm,
 top=20mm,
 }
\usepackage{stackengine}
\newcommand\ubar[1]{\stackunder[1.2pt]{$#1$}{\rule{.8ex}{.075ex}}}
\title{Analisi I}
\author{Luca Vettore}
\date{Primo semestre 2021}
\begin{document}
\maketitle

\section{Insiemi numerici}
Prima di cominciare a studiare i concetti dell'analisi matematica è necessario definire alcuni elementi di base, primi tra tutti gli insiemi numerici.

\subsection{Gli insiemi $\mathds{N}$ e $\mathds{Z}$}
Il primo insieme da considerare è l'insieme dei numeri naturali $\mathds{N}$. L'esistenza di questo insieme e le sue proprietà sono spesso postulate. $\mathds{N}$ è formato dai numeri interi non negativi (nella definizione più comune non include lo 0).\\\\
$\mathds{N}_0$ è definito come $\mathds{N} \cup \{0\}$.\\\\
L'insieme dei numeri interi relativi $\mathds{Z}$ include numeri interi positivi e negativi, oltre allo 0.\\
$\mathds{Z}$ ha alcune proprietà:
\begin{itemize}
    \item
    Possiede l'inverso della somma: $n + (-n) = 0$ $\forall n \in \mathds{Z}$
    \item
    $(\mathds{Z}, +)$ è u gruppo (l'addizione è associativa, commutativa e ha un inverso)
    \item
    $(\mathds{Z}, \cdot)$ non è un gruppo perché non è chiuso ($\exists n:$ $\frac{1}{n}$ $\notin$ $\mathds{Z}$)
\end{itemize}

\subsection{L'insieme $\mathds{Q}$}
$\mathds{Q}$ è l'insieme dei numeri razionali (\{$r=\frac{p}{q};$ $p,q \in \mathds{Z}$, $q \neq 0$\}). Ha le seguenti proprietà:
\begin{itemize}
    \item
    Possiede l'inverso della somma: $r + (-r) = 0$ $\forall r \in \mathds{Q}$
    \item
    $(\mathds{Q}, +)$ è u gruppo (l'addizione è associativa, commutativa e ha un inverso)
    \item
    $(\mathds{Q}, \cdot)$ è un gruppo perché è chiuso ($r \neq 0 \rightarrow r \cdot \frac{1}{r} = 1$)
    \item
    Ogni elemento può essere rappresentato da infinite coppie p e q, ma solo una in cui p e q sono coprimi
    \item
    Ogni elemento può essere rappresentato univocamente da un allineamento decimale del tipo $c_o, c_1 c_2 ... c_n$ con $n \in \mathds{N}$ e $n \leq q$
\end{itemize}
\textbf{Teorema: allineamenti decimali}\\
Ogni $r \in \mathds{Q}$ può essere rappresentato da un allineamento decimale con $n \in \mathds{N}$ e $n \leq q$.\\\\
\textbf{Dimostrazione:}
\begin{enumerate}
    \item
    Se denotiamo con $c_0$ la parte intera di r, $c_n$ la cifra ennesima e $p_n$ il resto ennesimo: $\frac{p}{q} = c_o + \frac{p_0}{q}$
    \item
    Allo stesso modo: $\frac{p_0}{q} = \frac{1}{10} \cdot \frac{10 p_0}{q} = \frac{1}{10} \cdot \left(c_1 + \frac{p_1}{q} \right) = \frac{c_1}{10} + \frac{p_1}{10q}$ \qquad $0 \leq c_1 \leq 9$ e $p_1 < q$
    \item
    Applicando ripetutamente lo stesso procedimento per i resti successivi otteniamo: $r = c_0 + \frac{c_1}{10} + \frac{c_2}{10^2} + \frac{c_3}{10^3} + ... + \frac{c_n}{10^n} + ...$ \\\\
    I resti possibili sono sempre $0 \leq p_n < q-1$, quindi dopo un massimo di q passaggi otteniamo resti uguali ai precedenti. In questo caso chiamiamo l'allineamento periodico e definiamo periodo la parte decimale che si ripete e antiperiodo la precedente.\\\\
    Gli allineamenti con periodo 9 risultano identici agli allineamenti con periodo 0 ($0,\bar{9} = 1$). Questo algoritmo non produce mai i numeri con periodo $\bar{9}$.

\end{enumerate}

\subsection{L'insieme $\mathds{R}$}
L'insieme $\mathds{Q}$ non contiene la totalità degli allineamenti decimali, non contiene ad esempio $1,101001000100001...$ (dove ogni 1 è preceduto sempre da un numero maggiore di 0). Alcuni dei numeri che incontriamo come risultati di comuni equazioni non possono essere rappresentati da un allineamento razionale.\\\\
\textbf{Teorema: irrazionalità di $\sqrt{2}$}\\
L'equazione $x^2 = 2$ non ha soluzioni in $\mathds{Q}$.\\\\
\textbf{Dimostrazione:}\\
Supponendo per assurdo che esista una coppia di numeri naturali p e q primi tra loro tali che $\frac{p}{q} = \sqrt{2}$ otteniamo che $2 = \frac{p^2}{q^2} \rightarrow p^2 = 2 q^2$, da cui deduciamo che p deve essere pari. $2m = p \rightarrow 4m^2 = 2p^2 \rightarrow 2m^2 = p^2$, quindi p è pari, ma p e q sono primi fra loro per ipotesi $\rightarrow$ assurdo.\\
\\
Un numero reale è definito come un allineamento decimale (periodico o no) preceduto da segno $\pm$. Per definizione $\mathds{N} \subset \mathds{Z} \subset \mathds{Q} \subset \mathds{R}$ e l'insieme $\mathds{R} \setminus \mathds{Q}$ è chiamato insieme dei numeri irrazionali.
\subsubsection{Ordinamento}
Siano $\alpha \in \mathds{R}$ e $\beta \in \mathds{R}$, $\alpha \neq \beta$ non negativi (una definizione analoga vale per negativi e discordi) e siano $a_n$ e $b_n$ le prime cifre diverse, allora se $a_n > b_n \rightarrow \alpha > \beta$ e viceversa.\\\\
L'ordinamento in $\mathds{R}$ ha le seguenti proprietà:
\begin{enumerate}
    \item
    $\forall \alpha, \beta \in \mathds{R}$ vale solo una delle relazioni $\alpha = \beta$ oppure $\alpha < \beta$ opure $\alpha > \beta$
    \item
    $\forall \alpha, \beta, \gamma \in \mathds{R}$ se $\alpha < \beta$ e $\beta < \gamma$ allora $\alpha < \gamma$
\end{enumerate}
Sia $A \subset \mathds{R}$ e $A \neq \emptyset$
\begin{itemize}
    \item 
    A è superiormente limitato se $\exists \Lambda \in \mathds{R}$: $\forall a \in A$ $a \leq \Lambda$
    \item
    A è inferiormente limitato se $\exists \lambda \in \mathds{R}$: $\forall a \in A$ $a \geq \lambda$
    \item
    A è limitato se valgono le due condizioni precedenti
\end{itemize}
Un numero reale M è detto massimo di un insieme $A \neq \emptyset$; $A \subset \mathds{R}$ se $M \in A$ e $\forall a \in A$ $M \geq a$ e si denota max A\\
Un numero reale m è detto minimo di un insieme $A \neq \emptyset$; $A \subset \mathds{R}$ se $m \in A$ e $\forall a \in A$ $m \leq a$ e si denota min A\\
Un sottoinsieme di $\mathds{R}$ non ha necessariamente massimo e minimo, ma se esistono questi sono necessariamente unici.\\\\
Un numero reale $\Lambda$ è maggiorante di $A \subset \mathds{R}$ se $\Lambda \in \mathds{R}$ e $\forall a \in A$ $\Lambda \geq a$\\
Un numero reale $\lambda$ è minorante di $A \subset \mathds{R}$ se $\lambda \in \mathds{R}$ e $\forall a \in A$ $\lambda \leq a$\\
Massimo e minimo sono rispettivamente maggioranti e minoranti di un insieme, ma un maggiorante o un minorante non sono necessariamente massimo e minimo.\\
Un insieme $A \subset \mathds{R}$; $A \neq \emptyset$ ha sicuramente maggioranti e minoranti, ma non massimo e minimo.\\\\
Sia $A \subset \mathds{R}$, se A è superiormente limitato è detto estremo superiore ($sup A$) il minimo dei maggioranti, se A è inferiormente limitato è detto estremo inferiore ($inf A$) il massimo dei minoranti.
($\Lambda=\sup A\Leftrightarrow\forall x\in A$ $\Lambda>x$ e $\forall\epsilon>0\exists x\in A:$ $L-\epsilon<x$.)\\
Estremo superiore e inferiore sono unici. Se A non è superiormente limitato per convenzione $sup A = +\infty$, allo stesso modo se A non è inferiormente limitato $inf A = -\infty$.\\\\
\textbf{Teorema: completezza di $\mathds{R}$}\\
Sia $A \subset \mathds{R}$ un insieme superiormente limitato, allora l'insieme dei maggioranti ha minimo. Se A è inferiormente limitato, allora l'insieme dei minoranti ha massimo.\\\\
\textbf{Dimostrazione:}\\
Ci limitiamo a dimostrare la prima condizione, con tutti i maggioranti $\geq 0$, gli altri casi sono analoghi.\\\\
Chiamiamo B l'insieme dei maggioranti di A. L'insieme delle parti intere di B ha minimo non negativo ($\leftarrow$ l'insieme parti intere di B è $\subset \mathds{N}$). Denotiamo con $c_0$ questo minimo e definiamo $B_0$ := $\{\beta \in B$: la parte intera di $\beta$ è $c_0 \}$.\\
L'insieme delle prime cifre decimali $c_1$ degli elementi di $B_0$ ha minimo ($\leftarrow c_1 \in \mathds{N}$ e $0 \leq c_1 \leq 9$). Definiamo quindi $B_1$ := $\{\beta \in B_0:$ la prima cifra decimale di $\beta$ è $c_1$\}.\\
Possiamo applicare lo stesso procedimento per $B_k$ := \{$\beta \in B_{k - 1}:$ la k-esima cifra di $\beta$ è $c_k$\}. $\forall k$ $B_k \neq \emptyset$ e $B_k \subseteq B_{k-1}$.\\
Poniamo $\gamma$ := $c_0, c_1 c_2 ... c_k ...$
\begin{itemize}
    \item $\gamma \in \mathds{R}$:\\
    Supponiamo per assurdo che $\gamma$ abbia periodo 9 $\rightarrow \gamma \notin \mathds{R}$. $c_k$ è minimo delle k-esime cifre decimali di $B_k$, quindi $c_k = 9$ $\rightarrow$ tutti gli elementi di $B_k$ hanno k-esima cifra = 9, ma allora $B_{k-1} = B_k = B_{k + 1} = ...$ e quindi tutti gli elementi di $B_{k-1}$ avrebbero periodo 9 che è assurdo.
    \item $\gamma$ è maggiorante di A:\\
    Supponiamo per assurdo che $\gamma$ non sia maggiorante di A. Allora $\exists \alpha \in A$: $\alpha > \gamma \rightarrow$ definito k il primo indice per cui $a_k \neq c_k$ $\alpha > \beta$ $\forall\beta \in B_k$
    \item $\gamma$ è minimo dei maggioranti:\\
    Sia $\beta \in B$, allora\\
    o $\beta \in B_k \in B$ $\forall k \in \mathds{N}_0 \rightarrow \beta = \gamma$\\
    oppure $\exists k \in \mathds{N}_0: \beta \notin B_k \rightarrow \beta > \gamma$
\end{itemize}
$\Rightarrow \gamma$ è $sup A$\\
\\
L'insieme $\mathds{Q}$ non è completo (A := \{ $r \in \mathds{Q}^+: r^2 < 2$\} è limitato, ma non esiste $sup A \in \mathds{Q}$).\\\\
\textbf{Teorema: densità di $\mathds{Q}$ e ($\mathds{R} \setminus \mathds{Q}$})\\
Tra due numeri reali esistono infiniti numerali razionali e irrazionali.\\\\
\textbf{Dimostrazione:}\\
Consideriamo il caso $x,y \in \mathds{R}$; $0<x<y$. Gli altri sono analoghi.
\begin{itemize}
    \item $\Rightarrow \exists n: x_n < y_n$
    \item $\Rightarrow \exists k>n: x_k<9$
\end{itemize}
$r=x_0,x_1 x_2 ... x_n ... x_{k-1} 9 \bar{0}$: \begin{itemize}
    \item $r \in \mathds{Q}$
    \item $r < y \leftarrow$ la prima cifra diversa è la n-esima e $x_n=r_n<y_n$
    \item $r > x \leftarrow$ la prima cifra diversa è la k-esima e $r_k=9>x_k$
\end{itemize}
$z=x_0,x_1 x_2 ... x_n ... x_{k-1} 9 0$ + [qualsiasi successione di numeri non periodica]: \begin{itemize}
    \item $z \notin \mathds{Q}$
    \item $z < y \leftarrow$ la prima cifra diversa è la n-esima e $x_n=r_n<y_n$
    \item $z > x \leftarrow$ la prima cifra diversa è la k-esima e $r_k=9>x_k$
\end{itemize}
Ripetendo questo procedimento all'infinito si ha la tesi.\\


\subsubsection{Troncamenti e operazioni}
Tutte le operazioni algebriche possibili in $\mathds{Q}$ si possono estendere a $\mathds{R}$. Per estenderle è però necessario definire il concetto di troncamento.\\
Il troncamento k-esimo di $x \in \mathds{R}$ è definito come: $x^{(k)}:=x_0,x_1x_2...x_k \bar{0}$, dove $sup \{x^{(k)}; k\in \mathds{N}\} = x$\\
A questo punto è possibile definire la somma di due numeri reali $\alpha$ e $\beta$ come: $\alpha + \beta := sup \{\alpha^{(k)}+\beta^{(k)}; k\in \mathds{N}\}$ e allo steso modo le altre operazioni.
\section{Le funzioni}
Il concetto di funzione può essere definito a partire dal prodotto cartesiano, ma per ora ci limitiamo a descriverle come una legge che dati due insiemi $X,Y \neq \emptyset$ associa ad ogni elemento di X uno e un solo elemento di Y.
\begin{itemize}
    \item $X$ è detto insieme di definizione o dominio.
    \item $Y$ è detto insieme di arrivo o codominio.
    \item $f(x)$ è detta immagine di $x$ tramite $f$.
    \item L'insieme $f(X):= \{f(x); x \in X \}$ è detta immagine di $f(x)$.
    \item L'insieme $graf(f) := \{(x,y) \in X \times Y: y=f(x); x \in X \}$ è deto grafico di $f$
    \item L'insieme $\{x \in X: f(x)=y \}$ è detta controimmagine di $f$
    \item L'insieme $f^{-1}(B) := \{x \in X: f(x) \in B \}$, con $B \subset Y$ è detta controimmagine di $B$ attraverso ad $f$
\end{itemize}
$f(X) \subset Y$ sempre, ma non necessariamente $f(X) = Y$.
\begin{itemize}
    \item Una funzione è detta suriettiva se $f(X) = Y$
    \item Una funzione è detta iniettiva se $\forall x_1,x_2 \in X, x_1 \neq x_2 \Rightarrow f(x_1) \neq f(x_2)$
    \item Una funzione è detta biettiva se è iniettiva e suriettiva
\end{itemize}
Se $f(x)$ è biettiva allora è definita la funzione inversa $f^{-1}(y):=x$.\\\\
Siano $f: X \rightarrow Y$ e $g: Y \rightarrow Z$ due funzioni, è detta funzione composta: $g o f(x):=g(f(x))$. La composizione di due funzioni biunivoche è anch'essa biunivoca.\\

\subsection{Le successioni}
Sia $X \neq \emptyset$ un insieme, una successioni a valori in $X$ è una qualunque funzione $f: \mathds{N} \rightarrow X.$ L'immagine $f(n)$ è generalmente denotata $x_n$. La successione può essere rappresentata dall'elenco dei suoi valori ($x_0,x_1,x_2,x_3,...,x_n,...$) oppure con la notazione  $\{x_n\}_{n \in N}$. L'insieme degli $n$ è chiamato insieme degli indici.\\\\
Data una successione $\{x_n\}_{n \in N}$ e una successione crescente di interi $n_1, n_2, n_3, ..., n_k, ...$, è detta sottosuccessione di $\{x_n\}_{n \in N}$ la successione $\{x_{nk}\}_{k \in N}$.
\subsection{La cardinalità}
Sia $X \neq \emptyset$, si dice che $X$ è finito se è in corrispondenza biunivoca con $\{1, 2, 3, ..., n\}$. Il numero $n$ è detta cardinalità o potenza di $X$. Un insieme non vuoto si dice infinito se non è finito.\\
Due insiemi infiniti $X,Y$ hanno la stessa cardinalità (o sono equipotenti) se $\exists (f: X \rightarrow Y)$ biunivoca e si denota $X \sim Y$.\\
$X \neq \emptyset$ è detto numerabile se $X \sim \mathds{N}$.\\\\
\textbf{Teorema}\\
Sia $X,A \neq \emptyset$, $X$ numerabile, $A$ infinito e $A \subset X$, allora $A$ è numerabile.\\\\
\textbf{Dimostrazione}\\
$X$ numerabile $\Rightarrow \exists (f: \mathds{N} \rightarrow X)$ biunivoca.\\
Poniamo: $a_1 = x_{k1}$ dove $k_1 = min\{k \in \mathds{N}: x_k \in A\}$; $a_2 = x_{k2}$ dove $k_2 = min\{k>k_1 \in \mathds{N}: x_k \in A\}$; ...\\
Per induzione troviamo una successione $a_n$, quindi una funzione $f: \mathds{N} \rightarrow A$ biunivoca $\Rightarrow A$ è numerabile.\\
\\
\textbf{Teorema}\\
Sia $A_j$ una successione di insiemi numerabili, allora \begin{itemize}
    \item $A:= \bigcup_{j \in \mathds{N}} A_j$ è numerabile
    \item $A^k:= \bigcup_{j = 1} ^k A_j$ è numerabile
\end{itemize}
\textbf{Dimostrazione}\\
Ogni $A_j$ è numerabile $\Rightarrow$ $\exists f_j: \mathds{N} \rightarrow A_j$ biunivoca, ponendo $a_{jn} := f_j(n)$, otteniamo che $A_j = \{a_{j1}, a_{j2}, ..., a_{jn}, ...\}$, da cui:\\
$A_1: a_{11}, a_{12}, a_{13}, a_{14}, a_{15}, ...$\\
$A_2: a_{21}, a_{22}, a_{23}, a_{24}, a_{25}, ...$\\
$A_3: a_{31}, a_{32}, a_{33}, a_{34}, a_{35}, ...$\\
Possiamo quindi definire una successione che elenchi tutti gli elementi dell'unione leggendo lungo le diagonali: $\bigcup_{j \in \mathds{N}} A_j = \{a_{11}, a_{21}, a_{12}, a_{31}, a_{22}, a_{13}, ...\}$\\\\
Abbiamo così una funzione biunivoca tra $\mathds{N}$ e $\bigcup_{j \in \mathds{N}} A_j$ semplicemente rinominando gli indici. Nel caso in cui ci fossero elementi presenti in più di un insieme sarebbe sufficiente eliminarli mano a mano mentre si applica lo stesso metodo.\\\\
$A^K := \bigcup_{j=1} ^K A_j$ è numerabile $\Leftarrow A^K \subset \bigcup_{j \in \mathds{N}} A_j$\\
\\
\textbf{Teorema}\\
Siano $A$ e $B$ due insiemi numerabili, allora il loro prodotto cartesiano $A \times B$ è numerabile.\\\\
\textbf{Dimostrazione}\\
$A$ numerabile $\Rightarrow$ posso definire una successione ${a_n}$ contenente tutti i suoi elementi, lo stesso vale per b.\\\\
$A \times B$:\\
$(a_1,b_1), (a_1,b_2), (a_1,b_2), (a_1,b_2), ...$\\
$(a_2,b_1), (a_2,b_2), (a_2,b_2), (a_2,b_2), ...$\\
$(a_3,b_1), (a_3,b_2), (a_3,b_2), (a_3,b_2), ...$\\\\
Come nel teorema precedente, è sufficiente seguire le diagonali per ottenere un ordinamento.\\
\\
\textbf{Teorema: numerabilità di $\mathds{Z}$ e $\mathds{Q}$}\\
$\mathds{Z}$ e $\mathds{Q}$ sono numerabili.\\\\
\textbf{Dimostrazione}\\
$\mathds{Z} = \mathds{N}_0 \cup -\mathds{N} \Rightarrow$ $\mathds{Z}$ è numerabile.\\\\
$\mathds{Q} = \mathds{Q}^+ \cup \{0\} \cup \mathds{Q}^-$ \begin{itemize}
    \item $\mathds{Q}^+ = \{\frac{p}{q}; p,q \in \mathds{N} \} \sim \mathds{N} \times \mathds{N} \Rightarrow$ numerabile
    \item $\mathds{Q}^- \sim \mathds{Q}^+$
\end{itemize}
$\Rightarrow \mathds{Q}$ è numerabile\\
\\
Un insieme è detto al più numerabile se è finito o numarabile.\\
Un insieme ha la potenza del continuo se è equipotente a $\mathds{R}$. Ogni intervallo in $\mathds{R}$ ha la potenza del continuo (ad es: $f(x)= \frac{x}{1-|x|}: (-1, 1) \rightarrow \mathds{R}$ biunivoca).\\\\
\textbf{Teorema: teorema di Cantor}\\
$\mathds{R}$ ha potenza maggiore del numerabile.\\\\
\textbf{Dimostrazione}\\
$(0, 1) \sim \mathds{R} \Rightarrow$ basta dimostrare che $\nexists f: \mathds{N} \rightarrow (0,1)$ biunivoca.\\
Supponiamo per assurdo che $\exists f: \mathds{N} \rightarrow (0,1)$ biunivoca, possiamo quindi elencare gli elementi di (0,1). Considero gli allineamenti decimali:\\
$x_1: 0,c_{11}c_{12}c_{13}c_{14}$\\
$x_2: 0,c_{21}c_{22}c_{23}c_{24}$\\
...\\
$x_n: 0,c_{n1}c_{n2}c_{n3}c_{n4}$\\
Posso sempre costruire un $x \in (0,1)$ non nell'elenco: \begin{itemize}
    \item $x = 0,c_1c_2c_3...c_n...$
    \item $c_1 \neq 0; c_1 \neq c_{11}; c_1 \neq 9$
    \item $c_2 \neq c_{22}; c_2 \neq 9$
    \item $c_n \neq c_{nn}; c_n \neq 9$
\end{itemize}
$x \in \mathds{R} \Leftarrow$ non ha periodo 9, $x \in (0,1)$, $x$ non è nell'elenco che è assurdo.\\
\\
\textbf{Corollario}\\
$\mathds{R} \setminus \mathds{Q}$ non è numerabile.

\section{Spazi euclidei}
Sia $n \in \mathds{N}$. Definiamo $\mathds{R}^n:= \mathds{R} \times \mathds{R} \times ... \times \mathds{R}$ (per n volte) $= \{(x_1,x_2,...,x_n): x \in \mathds{R}$ $\forall i=1,2,...,n\}$. Un elemento di $\mathds{R}^n$ si indica con $\ubar{x}$ e può essere rappresentato come un vettore n-dimensionale.\\
$\mathds{R}^n$ è uno spazio vettoriale su $\mathds{R}$, cioè al suo interno sono definite le operazioni vettoriali: \begin{itemize}
    \item $\ubar{x},\ubar{y} \in \mathds{R}^n$ \quad $\ubar{x} + \ubar{y}=\{x_1+y_1,...,x_n+y_n\}$
    \item $\ubar{x} \in \mathds{R}^n, \alpha \in \mathds{R}$ \quad $\alpha \cdot \ubar{x}=\{\alpha \cdot x_1,...,\alpha \cdot x_n\}$
\end{itemize}
Oltre a queste operazioni è anche possibile definire un prodotto interno (o prodotto scalare) con le seguenti proprietà: \begin{itemize}
    \item $<\ubar{x}, \ubar{y}>:=\sum_{i=1}^n x_iy_i$
    \item $<\ubar{x}, \ubar{y}> = <\ubar{y}, \ubar{x}>$
    \item $<\alpha \cdot \ubar{x}, \ubar{y}> = \alpha \cdot <\ubar{x}, \ubar{y}>$
    \item $<\ubar{x} + \ubar{y}, \ubar{z}> = <\ubar{x}, \ubar{z}> + <\ubar{y}, \ubar{z}>$
    \item $<\ubar{x}, \ubar{x}> \geq 0$
    \item $<\ubar{x}, \ubar{x}> = 0 \Leftrightarrow \ubar{x}=0$
\end{itemize}
Uno spazio $\mathds{R}^n$ dotato delle tre operazioni sopra citate è definito spazio euclideo.\\\\
Si definisce norma di un vettore $\ubar{x}$ il numero reale $||\ubar{x}|| = \sqrt{<\ubar{x}, \ubar{x}>}$\\\\
\textbf{Teorema: diseguaglianza di Cauchy-Schwartz}\\
$\forall \ubar{x},\ubar{y}\in\mathds{R}^n$ \quad $|<\ubar{x}, \ubar{y}>| \leq ||\ubar{x}||$ $||\ubar{y}||$\\\\
\textbf{Dimostrazione}\\
Se $\ubar{y} = 0$ la tesi è vera.\\
Altrimenti: $\forall \lambda \in \mathds{R}$ \quad $0\leq||\ubar{x}+\lambda\ubar{y}||^2=<\ubar{x}+\lambda\ubar{y},\ubar{x}+\lambda\ubar{y}>=<\ubar{x},\ubar{x}>+<\ubar{x},\lambda\ubar{y}>+<\lambda\ubar{y},\ubar{x}>+<\lambda\ubar{y},\lambda\ubar{y}>$ $=||x||^2+2\lambda<\ubar{x},\ubar{y}>+\lambda^2||y||^2$ che è un polinomio di secondo grado, fissati $\ubar{x}$ e $\ubar{y} \Rightarrow 0 \geq \Delta=4<\ubar{x},\ubar{y}>^2-4||\ubar{x}||^2||\ubar{y}||^2$
$\Rightarrow ||<\ubar{x},\ubar{y}>|| \leq ||x||$ $||y||$\\
\\
\textbf{Teorema: diseguaglianza triangolare}\\
$\forall\ubar{x},\ubar{y}\in\mathds{R}^n$ $||\ubar{x}+\ubar{y}||\leq||x||$ + $||y||$\\\\
\textbf{Dimostrazione}\\
$||\ubar{x}+\ubar{y}||^2\leq||x||^2||y||^2 \Rightarrow <\ubar{x},\ubar{x}>+<\ubar{x},\ubar{y}>+<\ubar{y},\ubar{x}>+<\ubar{y},\ubar{y}>=||\ubar{x}||^2+2<\ubar{x},\ubar{y}>+||y||^2 \leq ||\ubar{x}||^2||\ubar{y}||^2$\\
$\Rightarrow ||\ubar{x}+\ubar{y}||\leq||x||$ + $||y||$

\section{Spazi metrici}
Sia $X\neq\emptyset$ e $d: X\times X\rightarrow[0,+\infty]$ una funzione, $(X,d)$ è detto spazio metrico se $d$ rispetta le condizioni:
\begin{itemize}
    \item $d(x,y)=0 \Leftrightarrow x=y$
    \item $d(x,y)=d(y,x)$ $\forall x,y\in X$ (simmetria)
    \item $d(x,y)\leq d(x,z)+d(z,y)$ $\forall x,y,z\in X$ (diseguaglianza triangolare)
\end{itemize}
$d(x,y)$ è detta distanza dal punto $x$ al punto $y$.\\\\
Sia $(X,d)$ uno spazio metrico, è detta bolla aperta di raggio $r>0$ e centro $x\in X$ (o intorno) l'insieme\\ $B_r(x):=\{y\in X: d(x,y)<r\}$\\\\
\textbf{Teorema: proprietà di Hausdorff}\\
Sia $(X,d)$ spazio metrico, $x,y\in X$, $x\neq y$ allora $\exists r>0: B_r(x)\cap B_r(y)=\emptyset$\\\\
\textbf{Dimostrazione}\\
Poniamo $r=\frac{1}{3}d(x,y)$. Sia $z\in B_r(x)\Rightarrow$ per la disuguaglianza triangolare $3r=d(x,y)\leq d(x,z)+d(z,y)\leq r+d(y,z)$ $\Rightarrow d(y,z)\geq2r$ quindi $z\notin B_r(y)$\\

\subsection{Classificazione dei punti}
Sia $E\subset X$, un punto $p\in E$ si dice:
\begin{itemize}
    \item interno se $\exists r>0:B_r(p)\subset E$ ($p\in E$ necessariamente)
    \item esterno se $\exists r>0: B_r(p)\subset E^c$ ($p\notin E$ necessariamente)
    \item di frontiera se $\forall r>0$ $B_r(p)\cap E \neq \emptyset, B_r(p)\cap E^c \neq \emptyset$ ($p$ può appartenere a $E$ o $E^c$)
\end{itemize}
Un punto $p\in E$ è detto di accumulazione se $\forall r>0$ $\exists x\neq p:x\in B_r(p)\cap E$ oppure isolato se $\exists r>0:$ $B_r(p)\cap E=\{p\}$\\
Un punto di accumulazione non appartiene necessariamente all'insieme, un punto isolato si.\\\\
L'insieme $E^0$ è l'insieme dei punti interni di $E$, $\partial E$ l'insieme dei punti di frontiera e $E'$ l'insieme dei punti di accumulazione.\\\\
\textbf{Teorema}\\
Sia $(X,d)$ spazio metrico e $E\subset X, p\in E$, allora $p\in E'\Leftrightarrow\forall r>0$ $B_r(p)\cap E$ contiene infiniti punti di $E$.\\\\
\textbf{Dimostrazione}\\
Supponiamo per assurdo $p\in E'$ e $\exists r>0:$ $B_r(p)\cap E\setminus\{p\}=\{x_1,..,x_n\}, n\in\mathds{N}$\\
Poniamo $\bar{r}<min\{d(x,p), \forall x\in B_r(p)\cap E\setminus\{p\}\}$, allora $B_{\bar{r}}(p)\cap E=\{p\}$ che è assurdo.\\
\\
\textbf{Corollario}\\
Se $p\in E'$ allora $\exists\{x_n\}_{n\in\mathds{N}}\subset(E\setminus\{p\}):$ $d(x_n,p)<\frac{1}{n}$

\subsection{Insiemi limitati in $(X,d)$}
In uno spazio metrico generico $(X,d)$ non esiste necessariamente l'ordine, ma esiste la distanza. Definiamo quindi $diam(E):=sup_{x,y\in E}d(x,y)$. Si dice che $E\subset X$ è limitato se $diam(E)<+\infty$, altrimenti è illimitato.\\
Sia $E\in X, E\neq \emptyset$, $E$ è limitato $\Leftrightarrow\exists x_0\in X, \exists r>0:$ $E\subset B_r(x_0)$.\\
Ogni $E\subset X$ finito è limitato.\\
Se $E_1,E_2\subset X$ sono limitati allora $E_1 \cup E_2$ è limitato.

\subsection{Insiemi aperti e chiusi}
Sia $(X,d)$ uno spazio metrico, un insieme $E\subset X$ è detto:
\begin{itemize}
    \item aperto $\Leftrightarrow E=E^0$
    \item chiuso $\Leftrightarrow E^c$ è aperto
\end{itemize}
\textbf{Teorema:}\\
Sia $(X,d)$ uno spazio metrico, $E\subset X$ è chiuso $\Leftrightarrow E'\subset E$\\\\
\textbf{Dimostrazione}
\begin{itemize}
    \item $E$ chiuso $\Rightarrow E^c$ aperto. Sia $p\in E'$, supponiamo per assurdo che $p\notin E\Rightarrow\exists r>0:$ $B_r(p)\subset E^c\Rightarrow B_r(p)\cap E=\emptyset\Rightarrow p\notin E'$ che è assurdo.
    \item $E'\subset E \Rightarrow \forall p\in E^c$ si ha che $p\notin E'$, quindi $\exists r>0:$ $(B_r(p)\cap E)\setminus\{p\}=\emptyset$, ma $p\notin E\Rightarrow B_r(p)\subset E^c\Rightarrow p$ è interno a $E^c$.
\end{itemize}
Sia $E\subset X$, $(X,d)$ metrico, è detta chiusura di $E$ l'insieme $\bar{H}=E\cup E'$\\\\
Sia $(X,d)$ metrico, $E,F\subset X$
\begin{itemize}
    \item $\bar{E}$ è chiuso
    \item $F$ chiuso, $E\subset F\Rightarrow \bar{E}\subset F$
    \item $F$ chiuso $\Rightarrow$ $\bar{E}=E$
    \item $F\subset E\Rightarrow \bar{F}\subset\bar{E}$
\end{itemize}
\textbf{Teorema: unioni e intersezioni}\\
Sia $(X,d)$ metrico:
\begin{enumerate}
    \item se $\{E_i\}_{i\in I}$ è na famiglia di aperti allora $\bigcup_{i\in I}E_i$ è aperto
    \item se $\{E_1,..,E_n\}$ è una famiglia finita di aperti $\bigcap_{i=1}^N E_i$ è aperto
    \item se $\{E_i\}_{i\in I}$ è na famiglia di chiusi allora $\bigcap_{i\in I}E_i$ è chiuso
    \item se $\{E_1,..,E_n\}$ è una famiglia finita di aperti $\bigcup_{i=1}^N E_i$ è chiuso
\end{enumerate}
\textbf{Dimostrazione}\\
\begin{enumerate}
    \item Sia $E=\bigcup_{i\in I}E_i$ dove $E_i$ è aperto. Sia $x\in E\Rightarrow \exists i_o\in I: x\in E_{i0}$ (aperto) $\Rightarrow$ $\exists r>0:$ $B_r(x)\subset E_{i0}\subset E$
    \item Sia $E=\bigcap_{i=1}^N E_i, E_i$ aperto. Sia $x\in E\Rightarrow x\in E_i \forall i=1,..,N\Rightarrow\exists r_1,..,r_n>0: B_{r_i}(x)\subset E_i \forall i=1,...,N$. Sia $0<r=min\{r_1,...,r_N\} \Rightarrow B_{r_i}(x)\subset B_r(x) \forall i \Rightarrow E\subset B_r(x)$
    \item Per dimostrare i punti 3 e 4 utilizziamo un lemma (leggi di Morgan) (dim sul libro): \begin{itemize}
        \item $\left(\bigcup_{i\in I} E_i\right)^c=\bigcap_{i\in I}E_i^c$
        \item $\left(\bigcap_{i\in I} E_i\right)^c=\bigcup_{i\in I}E_i^c$
    \end{itemize}
    $\left(\bigcap_{i\in I} E_i\right)^c$ è aperto $\Leftarrow \bigcup_{i\in I}E_i^c$ chiuso
    \item $\left(\bigcup_{i\in I} E_i\right)^c$ è aperto $\Leftarrow \bigcap_{i\in I}E_i^c$ chiuso
\end{enumerate}

\section{Limiti di successioni}
\subsection{In spazi metrici}
Sia $(X,d)$ spazio metrico, $\{x_n\}_{n\in\mathds{N}}$ una successione a valori in $X$. Si dice che $x_n$ converge a $p\in X$ se $\forall\epsilon>0 \exists N=N(\epsilon):$ $\forall n>N$ $d(x_n,p)<\epsilon$. P è detto limite di $x_n$ e si scrive: $lim_{n\rightarrow+\infty}x_n=p$\\
$x_n\rightarrow p$ per $n\rightarrow+\infty\Leftrightarrow$ $\forall\epsilon>0$ $\exists N=N(\epsilon)\in\mathds{N}: \forall n>N$ $x_n \in B_\epsilon(p)$\\\\
Non tutte le successioni convergono (es: $x_n=(-1)^n$)\\\\
\textbf{Teorema: unicità del limite}\\
Sia $(X,d)$ spazio metrico e $\{x_n\}$ una successione convergente. Il limite di $x_n$ è unico.\\\\
\textbf{Dimostrazione}\\
Supponiamo per assurdo che $p,p'=lim_{n\rightarrow+\infty}x_n$. Per la proprietà di Hausdorff: $\exists r>0: B_r(p)\cap B_r(p')=\emptyset$, ma $\forall r>0$ $x_n\in B_r(p)$ e $x_n\in B_r(p')$ definitivamente, che è una contraddizione.
\\\\
\textbf{Teorema}\\
Se una successione è convergete, allora è limitata.\\\\
\textbf{Dimostrazione}\\
Fisso $\epsilon=1\rightarrow\exists N(1):\forall n>N$ $d(x_n,p)<1\Rightarrow\forall m,n>N(1)$ $d(x_n,x_m)\leq d(x_n,p)+d(x_m,p)\leq 2$, quindi $E_n=\{x_n;n>N\}$ è limitato e $E=\{x_n;n\in\mathds{N}\}=\{x_1,...,x_n\}\cup E_n$ è anch'esso limitato.\\
\\
Essere limitata è condizione necessaria, ma non sufficiente.\\\\
Una successione è detta di Cauchy se soddisfa la condizione:$\forall\epsilon>0$ $\exists N:\forall n,m>N$ $d(x_n,x_m)<\epsilon$\\
Uno spazio metrico è detto completo se ogni successione di Cauchy converge. $\mathds{R}$ è completo.\\\\
\textbf{Teorema: convergenza $\rightarrow$ Cauchy}\\
Una successione convergente è una successione di Cauchy.\\\\
\textbf{Teorema:}\\
Ogni successione di Cauchy è limitata.\\\\
\textbf{Teorema: convergenza sottosuccesioni:}\\
Sia $\{x_n\}$ una successione a valori in $(X,d)$ metrico, se $x_n\rightarrow p$ allora ogni sottosuccesione di $x_n$ converge p.\\\\
\textbf{Dimostrazione}\\
Sia $\{x_{nk}\}$ una sottosuccessione di $\{x_n\}$, poiché $n_k$ è una successione crescente di interi $\exists K_0:n_{k_0}\geq N(\epsilon)\Rightarrow$ [...]\\\\
\textbf{Teorema: convergenza e punti di accumulazione}\\
Sia $\{x_n\}$ una successione a valori in $(X,d)$ metrico, se $p\in X$ è punto di accumulazione per $E=\{x_;n\in N\}$ allora esiste una sottosuccessione di $x_n$ che converge a p.\\\\
\textbf{Dimostrazione}\\
Se p è di accumulazione allora $\forall r>0$ $B_r(p)$ contiene infiniti elementi.\\
Scelgo $r=1,1/2,...,1/k,...$ e $n_1\in B_1(p)\rightarrow d(p,x_{n_1})<1$, ..., $n_k\in B_{1/k}(p)\rightarrow d(p,x_{n_k})<1$ e ottengo una sottosuccessione convergente a p.

\subsection{In $\mathds{R}$}
Sia $\{x_n\}$ una successione a valori in $\mathds{R}$, usando la metrica euclidea, si dice che $x_n$ tende a $p$ oppure
$lim_{n\rightarrow+\infty}x_n=p\in\mathds{R}$\\
$\Leftrightarrow\forall\epsilon>0$ $\exists N=N(\epsilon):\forall n>N$ $|x_n-p|<\epsilon$\\\\
In $\mathds{R}$ si può definire anche la convergenza per eccesso o per difetto:\begin{itemize}
    \item $x_n$ converge per eccesso $\Leftrightarrow\forall\epsilon>0\exists N=N(\epsilon):\forall n>N$
    $ p\leq x_n <p+\epsilon$
    \item $x_n$ converge per difetto $\Leftrightarrow\forall\epsilon>0\exists N=N(\epsilon):\forall n>N$
    $ p+\epsilon<x_n \leq p$
\end{itemize}
Il limite di una successione se esiste è unico.\\
Si possono inoltre definire le successioni divergenti:\begin{itemize}
    \item una successione $x_n$ si dice convergente se $\exists p\in\mathds{R}: lim_{n\rightarrow+\infty}x_n=p$
    \item si dice divergente a $+\infty$ se $\forall M>0$ $\exists N=N(M)\in\mathds{N}: \forall n\geq N$ $x_n>M$ e si denota $lim_{n\rightarrow+\infty}x_n=+\infty$
    \item si dice divergente a $-\infty$ se $\forall M>0$ $\exists N=N(M)\in\mathds{N}: \forall n\geq N$ $x_n<-M$ e si denota $lim_{n\rightarrow+\infty}x_n=-\infty$
\end{itemize}
Una successione si dice regolare se ha limite e irregolare se non lo ha.

\subsubsection{Monotonia}
Sia $\{x_n\}$ una successione a valor in $\mathds{R}$:\begin{itemize}
    \item si dice monotona crescente se $x_n\leq x_{n+1}$ $\forall n\in\mathds{N}$
    \item si dice monotona decrescente se $x_n\geq x_{n+1}$ $\forall n\in\mathds{N}$
\end{itemize}
\textbf{Teorema}\\
Ogni funzione monotona è regolare, in particolare:\begin{itemize}
    \item se $x_n$ è crescente allora $lim_{n\rightarrow+\infty}=sup\{x_n;n\in\mathds{N}\}$
    \item se $x_n$ è decrescente allora $lim_{n\rightarrow+\infty}=inf\{x_n;n\in\mathds{N}\}$
\end{itemize}
\textbf{Dimostrazione}\\
Consideriamo solo il primo caso, l'altro è analogo.\\
Sia $E=\{x_n;n\in\mathds{N}\}$, allora ci sono due possibilità:\begin{itemize}
    \item E è superiormente limitato:\\
    Sia $L=sup(E)\Rightarrow\forall\epsilon>0$ $\exists N=N(\epsilon):$ $L-\epsilon<x_n\leq L$ dalla definizione, ma $x_n$ è crescente $\Rightarrow\forall n\geq N$ $L-\epsilon<x_n\leq L$ $\Rightarrow lim_{n\rightarrow+\infty}x_n=sup(E)$
    \item E non è superiormente limitato:\\
    $\forall M>0$ $\exists N=N(M): x_N>M$ dalla definizione, ma $x_n$ è crescente $\Rightarrow\forall n\geq N$ $x_n>x_n>M$\\ $\Rightarrow lim_{n\rightarrow+\infty}x_n=+\infty=sup(E)$
\end{itemize}
\textbf{Corollario}\\
Una successione monotona e limitata è regolare.

\subsubsection{Calcolo di limiti}
Esistono dei teoremi che forniscono risultati utili al calcolo dei limiti in $\mathds{R}$.\\\\
\textbf{Teorema: confronto per successioni divergenti}\\
Siano $\{x_n\}$ e $\{y_n\}$ due successioni tali che $x_n\leq y_n$ definitivamente, allora:\begin{itemize}
    \item se $x_n\rightarrow+\infty$ allora $y_n\rightarrow+\infty$
    \item se $y_n\rightarrow-\infty$ allora $x_n\rightarrow-\infty$
\end{itemize}
\textbf{Dimostrazione}\\
Dimostriamo solo il primo caso, il secondo è analogo.\\
$x_n\rightarrow+\infty\Leftrightarrow\forall M>0$ $\exists N=N(M):$ $\forall n>N$ $x_n>M$, ma $y_n\geq x_n\Rightarrow y_n\rightarrow+\infty$\\
\\
\textbf{Teorema: confronto per successioni convergenti}\\
Siano $x_n,y_n,z_n$ successioni a valori reali tali che $x_n\leq z_n\leq y_n$ definitivamente, se $x_n\rightarrow l\in\mathds{R}$ e $x_n\rightarrow l\in\mathds{R}$ allora $x_n\rightarrow l$\\\\
\textbf{Dimostrazione}\\
La dimostrazione è analoga alla precedente\\\\
\textbf{Teorema: permanenza del segno}\\
Sia $\{x_n\}$ una successione a valori reali, se $x_n\rightarrow p$, $p>0$ allora $x_n>0$ definitivamente.\\\\
\textbf{Dimostrazione}\\
Dalla definizione $x_n\rightarrow p \Rightarrow \forall\epsilon>0\exists N=N(\epsilon):$ $p-\epsilon<x_n$ $\forall n>N$, scegliendo $\epsilon$ tale che $p-\epsilon>0$ e un $N$ appropriato ho la tesi.\\
\\
Per le successioni convergenti si possono dimostrare alcune proprietà algebriche:
\begin{itemize}
    \item $a_n+b_n\rightarrow a+b$, dove $a$ e $b$ sono i limiti di $a_n$ e $b_n$
    \item $\frac{a_n}{b_n}\rightarrow\frac{a}{b}$ con $b\neq0$
    \item $a_n\cdot b_n=a\cdot b$
\end{itemize}
Risultati simili si possono ricavare per successioni divergenti, con degli accorgimenti:
\begin{itemize}
    \item $a_n\rightarrow+\infty$, $b_n\rightarrow b\Rightarrow a_n+b_n\rightarrow+\infty$
    \item $a_n\rightarrow-\infty$, $b_n\rightarrow b\Rightarrow a_n+b_n\rightarrow-\infty$
    \item $a_n\rightarrow+\infty$, $b_n\rightarrow b>0\Rightarrow a_n\cdot b_n\rightarrow+\infty$
    \item $a_n\rightarrow+\infty$, $b_n\rightarrow b<0\Rightarrow a_n\cdot b_n\rightarrow-\infty$
    \item $a_n\rightarrow\pm\infty$, $\Rightarrow \frac{1}{a_n}\rightarrow0$
    \item $a_n\rightarrow\pm0, a_n>0$, $\Rightarrow \frac{1}{a_n}\rightarrow+\infty$
    \item $a_n\rightarrow\pm0, a_n<0$, $\Rightarrow \frac{1}{a_n}\rightarrow-\infty$
\end{itemize}
Restano escluse tutte le forme riconducibili a $+\infty-\infty$, $\pm\infty\cdot0$, $\frac{0}{0}$, $\frac{\pm\infty}{\pm\infty}$ che sono dette forme di indecisione e andranno valutate caso per caso.\\\\
\textbf{Teorema: criterio del rapporto per successioni}\\
Sia $\{x_n\}$ una successione a valori reali tale che $x_n>0$, se $\exists lim_{n\rightarrow+\infty}\frac{x_{n+1}}{x_n}=l\in[0,+\infty)$ allora:
\begin{itemize}
    \item $0\leq l<1\Rightarrow x_n\rightarrow 0^+$
    \item $1<l\leq+\infty\Rightarrow x_n\rightarrow+\infty$
\end{itemize}
\textbf{Dimostrazione}\\
Consideriamo il primo caso.\\
Dall'ipotesi ricaviamo che $\forall\epsilon>0\exists N(\epsilon)$: $l-\epsilon<\frac{x_{n+1}}{x_n}<l+\epsilon$, da cui $x_{n+1}<(l+\epsilon)x_n$.\\
Per iterazione posso ottenere che $x{N+k}<(l+\epsilon)^kx_N$ $\forall k\geq1$, ponendo $n=N+k$ $0<x_N<(l+\epsilon)^n(l+\epsilon)^{-N}x_N$, dove all'aumentare di $n\in\mathds{N}$ $(l+\epsilon)^{-N}x_N$ rimane costante e $(l+\epsilon)^n\rightarrow 0$, per confronto quindi $x_n\rightarrow 0$.\\
Nel secondo caso basta porre $y_n=\frac{1}{x_n}$ e verifico le ipotesi della dimostrazione precedente.\\
\\
Se $lim\frac{x_n}{y_n}=0$ si denota $y_n>>x_n$ o $x_n<<y_n$.\\
Usando questa notazione è possibile definire una "gerarchia degli infiniti": $(log_bn)^q<<n^p<<a^n<<n!<<n^n$ ($b>1,q>0,p>0,a>1$)\\\\
Siano $x_n$ e $y_n$ due successioni a valori reali, entrambe diverse da 0, si denota $x_n=o(y_n)$ se $\frac{x_n}{y_n}\rightarrow0$ o $x_n\sim y_n$ se $\frac{x_n}{y_n}\rightarrow1$. Se $x_n\sim y_n$ allora $x_n=y_n+o(y_n)$
\begin{itemize}
    \item $x_n=o(y_n)\Rightarrow c\cdot x_n=o(y_n)$, $c\neq0$
    \item $x_n=o(y_n)\Rightarrow x_n\cdot z_n=o(z_n\cdot y_n)$
    \item $x_n\sim y_n$ e $z_n=o(x_n)\Rightarrow a_n=o(y_n)$
    \item $x_n=o(z_n)$ e $y_n=o(z_n)\rightarrow x_n\pm y_n=o(z_n)$
\end{itemize}

\section{Serie a valori reali}
Sia $\{a_n\}$ una successione a valori reali, definisco la somma parziale $S_k:=a_1+a_2+...+a_k$, il limite per $k\rightarrow+\infty$ della successione delle somme parziali è detto serie a valori reali di termine generale $a_n$ e corrisponde alla somma di tutti gli elementi di $a_n$ e si denota $\sum_{n=n_0}^{+\infty}a_n$.\\\\
La serie si dice convergente se la successione delle somme parziali converge, divergente se diverge oppure irregolare se non esiste il limite.
Calcolare le somme di infiniti termini in molti casi risulta difficile ed è spesso sufficiente studiarne il carattere (convergenza, ecc..) utilizzando appositi teoremi e riconducendosi a serie a carattere noto.

\subsection{Teoremi sul carattere delle serie}
\textbf{Teorema: Criterio di Cauchy per le serie}\\
La serie $a_n$ converge se e solo se rispetta la condizione di Cauchy: $\forall\epsilon>0\exists N=N(\epsilon)\in\mathds{N}:$ $\forall m\geq N$ $\forall k\geq N$ $\left|\sum_{n=m}^{m+k}a_n\right|<\epsilon$\\\\
\textbf{Dimostrazione}\\
$\sum_{n=m}^{m+k}a_n=S_{m+k}-S_{m-1}$, dove $S_t$ è la somma parziale fino a $t$.\\
Ponendo $p=m-1$ e $q=m+k$ possiamo riscrivere la condizione di Cauchy come $(q>p)$ $\forall\epsilon>0 \exists N=N(\epsilon)$: $\forall q>p\geq N-1$ $|S_q-S_p|<\epsilon$, che è la condizione di Cauchy per la successione delle somme parziali e in $\mathds{R}$ è condizione necessaria e sufficiente alla convergenza.\\
\\
\textbf{Teorema: condizione necessaria, ma non sufficiente}\\
Se una serie comverge, il suo termine generale tende a 0.\\\\
\textbf{Dimostrazione}\\
Sia $S=\sum_{n=m}^{m+k}a_n$, $s\in\mathds{R}$ allora $S_k\rightarrow S\Rightarrow (S_k-S_{k-1})\rightarrow S-S=0$\\
\\
\textbf{Teorema: serie definitivamente uguali}\\
Due serie definitivamente uguali hanno lo stesso carattere\\\\
\textbf{Teorema: prodotto per costante}\\
Le serie $\sum_{n=n_0}^{+\infty}a_n$ e $\sum_{n=n_0}^{+\infty}c\cdot a_n$ hanno lo stesso carattere.

\subsubsection{Serie a termine generale non negativo}
I seguenti teoremi si applicano solo nel caso in cui il termine generale della serie sia maggiore o uguale a 0. Le somme parziali di una serie a termine generale non negativo costituiscono una successione non decrescente.\\\\
\textbf{Teorema}\\
Sia $a_n\geq0$ allora $\sum_{n=n_0}^{+\infty}a_n$ è regolare.\\\\
\textbf{Teorema: criterio del confronto per le serie}\\
Siano $\sum_{n=n_0}^{+\infty}a_n$ e $\sum_{n=n_0}^{+\infty}b_n$ due serie tali che $0\leq a_n\leq b_n \forall n\in\mathds{N}$, allora \begin{enumerate}
    \item se $\sum_{n=n_0}^{+\infty}b_n$ converge anche $\sum_{n=n_0}^{+\infty}a_n$ converge
    \item se $\sum_{n=n_0}^{+\infty}a_n$ diverge anche $\sum_{n=n_0}^{+\infty}b_n$ diverge
\end{enumerate}
\textbf{Dimostrazione}
\begin{enumerate}
    \item Ponendo $S_k:=\sum_{n=n_0}^{k}a_n$ e $T_k=\sum_{n=n_0}^{k}b_n$ si ha $0\leq S_k\leq T_k$. Chiamando $\sum_{n=n_0}^{+\infty}b_n=T\in\mathds{R}$, $T$ è maggiorante di $\{S_k\}$, quindi $S_k$ è limitata e crescente, quindi converge.
    \item E sufficiente applicare il teorema del confronto alle successioni delle somme parziali
\end{enumerate}
\textbf{Teorema: criterio asintotico}\\
Siano $\sum_{n=n_0}^{+\infty}a_n$ e $\sum_{n=n_0}^{+\infty}b_n$ due serie a termine generale non negativo, tali che $a_n\sim b_n$, allora le due serie hanno lo stesso carattere.\\\\
\textbf{Dimostrazione}\\
Per ipotesi $lim_{n\rightarrow+\infty}\frac{a_n}{b_n}=1$, quindi $\forall\epsilon>0$ $1-\epsilon\leq\frac{a_n}{b_n}\leq\epsilon+1$\\
fissato un $\epsilon^*>0$ si ha $0<(1-\epsilon^*)\cdot b_n<a_n<(\epsilon^*+1)\cdot b_n$ definitivamente e posso quindi applicare il teorema del confronto.\\
\\
\textbf{Teorema: criterio del rapporto per le serie}\\
Sia $\sum_{n=n_0}^{+\infty}a_n$ una serie a termine generale non negativo tale che $\lim_{n\rightarrow+\infty}\frac{a_{n+1}}{a_n}=l\in[0,+\infty]$ allora:\begin{itemize}
    \item se $0\leq l<1$ la serie converge
    \item se $1<l\leq+\infty$ la serie diverge
    \item se $l=1$ non si può dire nulla sul carattere della serie
\end{itemize}
\textbf{Teorema: criterio della radice}\\
Sia $\sum_{n=n_0}^{+\infty}a_n$ una serie a termine generale non negativo tale che $\lim_{n\rightarrow+\infty}\sqrt[n]{a_n}=l\in[0,+\infty]$ allora:\begin{itemize}
    \item se $0\leq l<1$ la serie converge
    \item se $1<l\leq+\infty$ la serie diverge
    \item se $l=1$ non si può dire nulla sul carattere della serie
\end{itemize}
\textbf{Teorema: criterio di condensazione}\\
Sia $a_n$ una successione decrescente e positiva, le serie $\sum_{n=1}^{+\infty}a_n$ e $\sum_{n=1}^{+\infty}2^na_{2^n}$ hanno lo stesso carattere. Quest'ultima è detta serie condensata.

\subsubsection{Altri criteri di convergenza}
Nel caso di serie a termine generale negativo è sufficiente moltiplicare per $c=-1$ e applicare i teoremi precedenti. Per serie il cui termine generale non è definitivamente positivo o negativo esistono alcuni criteri che possono essere applicati.\\\\
\textbf{Teorema: convergenza assoluta}\\
Se la serie $\sum_{n=1}^{+\infty}|a_n|$ converge, allora anche $\sum_{n=1}^{+\infty}a_n$ converge.\\\\
\textbf{Dimostrazione}\\
Definiamo parte positiva di $a_n$ come $a_n^+=\max(a_n,0)$ e parte negativa di $a_n$ come $a_n^-=\max(-a_n, 0)$. [nota: $a_n^-$ è sempre positivo!]\\
$a_n=a_n^+-a_n^-$ e $|a_n|=a_n^++a_n^-$, quindi $0\leq a_n^+\leq|a_n|$ e $0\leq a_n^-\leq|a_n|$ $\forall n$, quindi se $\sum_{n=1}^{+\infty}|a_n|$ converge, per confronto anche $\sum_{n=1}^{+\infty}a_n^+$ e $\sum_{n=1}^{+\infty}a_n^-$ convergono e quindi la serie $\sum_{n=1}^{+\infty}a_n=\sum_{n=1}^{+\infty}a_n^+-\sum_{n=1}^{+\infty}a_n^-$ converge.\\
\\
Una serie si dice assolutamente convergente se la serie dei moduli converge.\\\\
\textbf{Teorema: criterio di Leibniz}\\
Sia $\sum_{n=1}^{+\infty}(-1)^na_n$ una serie tale che:
\begin{itemize}
    \item $a_n>0$ $\forall n$
    \item $a_n>a_{n+1}$ $\forall n$
    \item $\lim_{n\rightarrow+\infty}a_n=0$
\end{itemize}
la serie converge e in particolare se denotiamo con S la somma della serie e con $S_m$ la sua somma parziale $|S-S_m|\leq a_{m+1}$\\\\
\textbf{Dimostrazione}\\

\subsubsection{Serie notevoli}
Esistono alcune serie con carattere (e a volte somma) noto che possono risultare utili.
\begin{itemize}
    \item \textbf{Serie di Mengoli:} $\sum_{n=1}^{+\infty}\frac{1}{n(n+1)}=1$
    \item \textbf{Serie geometrica:} $\sum_{n=0}^{+\infty}q^n$ \begin{itemize}
        \item se $q\geq1$ diverge a $+\infty$
        \item se $|q|<1$ converge a $\frac{1}{1-q}$
        \item se $q\leq-1$ è irregolare
    \end{itemize}
    \item \textbf{Serie armonica:} $\sum_{n=1}^{+\infty}\frac{1}{n}$ diverge a $+\infty$
    \item \textbf{Serie armonica generalizzata:} $\sum_{n=2}^{+\infty}\frac{1}{n^p\log^qn}$ \begin{itemize}
        \item converge per \begin{itemize}
            \item $p>1$ e $q\in\mathds{R}$
            \item $p=1$ e $q>1$
        \end{itemize}
        \item diverge per \begin{itemize}
            \item $p=1$ e $q\leq1$
            \item $p<1$ e $q\in\mathds{R}$
        \end{itemize}
    \end{itemize}
    \item \textbf{Serie telescopica di passo uno:} $\sum_{n=1}^{+\infty}(b_{n+1}-b_n)=l-b_1$ con $b_n\rightarrow l$
\end{itemize}

\subsection{Riordinamenti}
In alcuni casi cambiare l'ordine dei termini non cambia la somma, mentre in altri si.\\
Definiamo un riordinamento della serie $\sum a_n$ come $\sum a_{\pi(n)}$, dove $\pi(n)$ è una funzione biunivoca $\mathds{N}\rightarrow\mathds{N}$.\\\\
\textbf{Teorema: convergenza incondizionata}\\
Se $\sum_{n=1}^{+\infty}a_n$ converge assolutamente, allora $\forall\pi(n)$ $\sum_{n=1}^{+\infty}a_n=\sum_{n=1}^{+\infty}a_{\pi(n)}$\\\\
\textbf{Dimostrazione}\\
Sia $S=\sum_{n=1}^{+\infty}a_n$, $b_n=a_{\pi(n)}$, $A_k=\sum_{n=1}^{k}a_n$ e $B_k=\sum_{n=1}^{+\infty}b_n$\\
Consideriamo il caso in cui $a_n\geq0$ $\forall n$, negli altri consideriamo parte positiva e negativa e procediamo allo stesso modo.\\
$\forall k\in\mathds{N}$ $B_k=\sum_{n=1}^{k}b_n\leq S$ (la somma di tutti gli elementi positivi della successione sarà sempre maggiore della somma di una parte degli elementi).\\
$B_k$ è quindi crescente e superiormente limitata da $S$, quindi $\lim B_k=B\leq S$.\\
Considero $\sum_{n=1}^{+\infty}a_n$ come riordinamento di $\sum_{n=1}^{+\infty}b_n$ e con un ragionamento analogo ottengo che $\lim A_k=S\leq B$, da cui $S=B$.\\
\\
\textbf{Teorema: di Riemann}\\
Se $\sum_{n=1}^{+\infty}a_n$ converge, ma non assolutamente allora $\forall S\in\bar{\mathds{R}}$ $\exists\pi:$ $\sum_{n=1}^{+\infty}a_{\pi(n)}=S$ e $\exists\pi$ tale che la serie risulti irregolare.

\section{Limiti di funzioni}

E' possibile estendere il concetto di limite dalle successioni alle funzioni in spazi metrici con definizioni simili.
\subsection{Funzioni in spazi metrici}
Siano $(X_1,d_1)$ e $(X_2,d_2)$ spazi metrici e $E\subset X_1, E\neq\emptyset$. Siano $p\in E'$ e $l\in X_2$, allora $\lim_{x\rightarrow p}f(x)=l$\\
$\Leftrightarrow\forall\epsilon>0$ $\exists\delta(\epsilon)>0:$ $\forall x\in E$ con $0<d_1(x,p)<\delta$ si ha che $d_2(f(x),l)<\epsilon$\\
$\Leftrightarrow\forall\epsilon>0$ $\exists\delta(\epsilon)>0:$ $\forall x\in [E\cap(B_r(p)\setminus\{p\})]$ si ha $f(x)\in b_\epsilon(l)$\\
$\Leftrightarrow\forall\epsilon>0$ $\exists\delta(\epsilon)>0:$ $f(E\cap(B_r(p)\setminus\{p\}))\subset B_\epsilon(l)$\\\\
\textbf{Teorema}\\
Siano $f:[E\subset X_1]\rightarrow X_2$, $p\in E'$ e $l\in X_2$,\\
allora $\lim_{x\rightarrow p}f(x)=l\Leftrightarrow\lim_{n\rightarrow+\infty}f(x_n)=l$ $\forall[\{x_n\}\subset (E\setminus\{p\}):$ $x_n\rightarrow p]$\\\\
\textbf{Dimostrazione}
\begin{itemize}
    \item Sia $\{x_n\}\subset (E\setminus\{p\}):$ $x_n\rightarrow p$\\
    $\lim_{x\rightarrow p}f(x)=l\Leftrightarrow\forall\epsilon>0$ $\exists\delta(\epsilon)>0:$ $\forall x\in E$ con $0<d_1(x,p)<\delta$ si ha che $d_2(f(x),l)<\epsilon$\\
    $x_n\rightarrow p\Rightarrow$ per qualsiasi $\delta(\epsilon)$ $\exists N(\delta(\epsilon)):$ $\forall n\geq N$ $0<d_1(x_n,p)<\delta$\\
    $\Rightarrow$ $d_2(f(x_1),l)<\epsilon$ $\forall n\geq N(\delta(\epsilon))\Rightarrow\lim_{x\rightarrow p}f(x)=l$
\end{itemize}
\textbf{Teorema: unicità}\\
Il limite di una funzione se esiste è unico
\subsection{Funzioni $\mathds{R}\rightarrow\mathds{R}$}
Ponendo $X_1=X_2=\mathds{R}$ e utilizzando la metrica euclidea la definizione di limite diventa:\\
$\lim_{x\rightarrow p}f(x)=l\Leftrightarrow\forall\epsilon>0$ $\exists\delta(\epsilon)>0:$ $\forall[x\in E:$ $0<|x-p|<\delta]\Rightarrow|f(x)-l|<\epsilon$\\\\
$+\infty$ e $-\infty$ sono punti di accumulazione per $\mathds{R}$, si possono quindi definire i limiti all'infinito come:\\
$\lim_{x\rightarrow+\infty}f(x)=l\in\mathds{R}\Leftrightarrow\forall \epsilon>0$ $\exists M(\epsilon)>0:$ $\forall x\in E$ con $x>M\Rightarrow|f(x)-l|<\epsilon$\\\\
e i limiti infiniti:\\
$\lim_{x\rightarrow+\infty}f(x)=l\Leftrightarrow\forall M>0$ $\exists\delta(M):$ $\forall x\in E$ con $0<|x-p|<\delta\Rightarrow f(x)>M$\\\\
Grazie alle proprietà di ordine e completezza i seguenti teoremi validi per le successioni valgono anche per le funzioni $\mathds{R}\rightarrow\mathds{R}$:
\begin{itemize}
    \item monotonia $\rightarrow$ regolarità
    \item teoremi del confronto
    \item algebra dei limiti
\end{itemize}

\section{Continuità}
Siano $(X_1,d_1)$ e $(X_2,d_2)$ spazi metrici, una funzione $f:[E\subset(X_1,d_1)]\rightarrow(X_2,d_2)$ è detta continua in un punto $p\in E$:\\
$\Leftrightarrow\forall\epsilon>0$ $\exists\delta(\epsilon,p):$ $\forall x\in E$ con $d_1(x,p)<\delta\Rightarrow d_2(f(x),f(p))<\epsilon$\\
$\Leftrightarrow\forall\epsilon>0$ $\exists\delta(\epsilon,p):$ $\forall x\in[E\cap B_\delta(p)]\Rightarrow f(x)\in B_\epsilon(f(p))$\\\\
$f$ continua in $p\Leftrightarrow\lim_{x\rightarrow p}f(x)=f(p)$.\\
Tutte le funzioni elementari sono continue sul loro dominio.
\subsection{In spazi metrici}
\textbf{Teorema; continuità della funzione composta}\\
Siano $(X_1,d_1)$, $(X_2,d_2)$, $(X_3,d_3)$ spazi metrici. Siano $E\subset X_1$ e $p\in E$. Siano $f:E\rightarrow X_2$ e $g:X_2\rightarrow X_3$ continue in $p$, allora la funzione composta $gof:E\rightarrow X_3$ è continua in $p$.\\\\
\textbf{Dimostrazione}\\
Se $p$ è isolato la funzione composta è continua.\\
Se $p\in E'$, considero una successione $\{x_n\}$ tale che $\{x_n\}\in E$ e $x_n\rightarrow p$. Per la continuità di $f$ $f(x_n)\rightarrow f(p)$ e per la continuità di $g$ $g(f(x_n))\rightarrow g(f(p))$, da cui la tesi attraverso il limite.\\
\\
\textbf{Teorema: continuità globale}\\
$f:(X_1,d_1)\rightarrow(X_2,d_2)$ è continua in $X_1\Leftrightarrow\forall A\subset X_2$ aperto $f^{-1}(A)$ è aperto.\\\\
\textbf{Dimostrazione}
\begin{itemize}
    \item Sia $f$ continua in $X_1$. Sia $A\subset X_2$ aperto. Sia $q\in A$ e $f(p)=q$.\\
    Scelto un $\epsilon>0$ per la continuità di $f$ esiste un $\delta>0$: $f(B_\delta(p))\subset B_\epsilon(q)$, da cui $B_\delta(p)\subset f^{-1}(B_\epsilon(q))\subset f^{-1}(A)$, da cui l'apertura di $A$ (tutti i punti sono interni)
    
    \item Sia $f^{-1}(A)$ aperto in $X_1$ per ogni $A$ aperto in $X_2$. Sia $p\in X_1$ e $f(p)=q$.\\
    Scelto un $\epsilon>0$ e posto $A=B_\epsilon(q)$ per ipotesi $f^{-1}(B_\epsilon(q))$ è aperto, quindi $\exists\delta>0:$ $B_\delta(p)\subset f^{-1}(B_\epsilon(q))$, da cui $f(b_\delta(p))\subset B_\epsilon(q)$, quindi $f$ continua in $p$ $\forall p\in X_1$
\end{itemize}

\subsection{In $\mathds{R}$}
Le funzioni reali continue hanno alcune proprietà particolari.\\\\
\textbf{Teorema: degli zeri}\\
Sia $f:[a,b]\rightarrow\mathds{R}$ con $b>a$ continua, $f(a)\cdot f(b)<0$, allora $\exists z\in(a,b):$ $f(z)=0$\\\\
\textbf{Teorema: valori intermedi}\\
Sia $f:[a,b]\rightarrow\mathds{R}$ con $b>a$ continua, essa assume tutti i valori compresi tra $f(a)$ e $f(b)$\\\\
\textbf{Dimostrazione}\\
Consideriamo il caso $f(a)<f(b)$, gli altri sono analoghi. \\
Sia $y_0\in(f(a),f(b))$, definiamo $g(x)=f(x)-y_0$, $g(x)$ è continua su $(a,b)$.\\
$g(a)<0$ e $g(b)>0$, quindi $g^{-1}(y_0)=0$ e si applica il teorema degli zeri. Questo procedimento si può applicare $\forall y_0\in(f(a),f(b))$.\\
\\
\textbf{Corollario}\\
$f:I\subset\mathds{R}\rightarrow\mathds{R}$, $I$ intervallo e $f$ continua in $I$, allora $f(I)$ è un intervallo.\\\\
\textbf{Teorema: proprietà algebriche della continuità}\\
Siano $f,g:E\subset(X,d)\rightarrow\mathds{R}$ continua in $x_0\in E$ e sia $(X,d)$ metrico, allora:
\begin{itemize}
    \item $f\pm g$ è continua in $x_0$
    \item $f\cdot g$ è continua in $x_0$
    \item $\frac{f}{g}$ è continua in $x_0:g(x_0)\neq0$
\end{itemize}
Sia $f:(a,b)\subset\mathds{R}\rightarrow\mathds{R}$ una funzione non continua in $x_0\in(a,b)$, allora se:
\begin{itemize}
    \item $\exists\lim_{x\rightarrow x_0^+}f(x)=\lim_{x\rightarrow x_0^-}f(x)\neq f(x_0)$ la discontinuità è di prima specie o eliminabile
    \item $\exists\lim_{x\rightarrow x_0^-}f(x)=l_1\in\mathds{R}, \exists\lim_{x\rightarrow x_0^+}f(x)=l_2\in\mathds{R}, l_1\neq l_2$ la discontinuità è di seconda specie o a salto
    \item Se almeno uno dei due limiti è infinito o non esiste la discontinuità è di terza specie
\end{itemize}
\textbf{Teorema:}\\
Sia $f:(a,b)\subset\mathds{R}\rightarrow\mathds{R}$ monotona, essa ha al più un'infinità numerabile di discontinuità di prima specie e nessuna delle altre.\\\\
\textbf{Teorema: continuità dell'inversa}\\
Siano $i,J\subset\mathds{R}$ intervalli e $f:I\rightarrow J$ continua e biunivoca, allora $f^-1:J\rightarrow I$ è continua.

\subsection{Compattezza}
Per definire la proprietà di compattezza è necessario prima definire il concetto di copertura aperta.\\
Una copertura aperta di un insieme generico $E$ è una famiglia di insiemi aperti $\{A_j\}_{j\in I}$ tali che $E\subset\bigcup_{j\in I}A_j$\\
Una famiglia finita di insiemi aperti estratta da una copertura aperta che verifica la medesima condizione è detta sottocopertura finita\\\\
Sia $(X,d)$ metrico, $E\subset X,E\neq\emptyset$ è detto compatto se da ogni sua copertura aperta si può estrarre una sottocopertura finita ($\Leftrightarrow\forall\{A_j\}:E\subset\bigcup_{j\in I}A_j$ $\exists\{A_{i_1},...,A_{i_N}\}:E\subset\bigcup_{i=1}^{N}A_{i_k}$)\\\\
\textbf{Teorema: condizione necessaria alla compattezza}\\
Sia $(X,d)$ metrico, $E\subset X,E\neq\emptyset$ compatto $\Rightarrow$ \begin{enumerate}
    \item E è chiuso e limitato
    \item ogni insieme infinito contenuto in E ha almeno un punto di accumulazione
\end{enumerate}
\textbf{Dimostrazione}
\begin{enumerate}
    \item Considero la famiglia $\{B_1(p)\}_{p\in E}$. Per la compattezza di $E$ $\exists\{p_1,...,p_N\}:$ $E\subset \bigcup_{k=1}^NB_1(P_k)$, ma $B_1(p_k)$ è limitato e chiuso $\forall k$, quindi $E$ è chiuso e limitato perché unione finita di insiemi chiusi e limitati.
    
    \item Sia $A\subset E$ infinito. Supponiamo per assurdo $A'=\emptyset$. Considero la copertura di $E$ $\{B_{r(p)}(p)\}_{p\in E}$ con $r(p): E\rightarrow\mathds{R}$. Dalla compattezza di $E$ $A\subset E\subset\bigcup_{k=1}^NB_{r(p)}(p_k)$, quindi $E$ contiene un numero finito di elementi di $A$, ma $A$ è infinito e contenuto in $E$ che è assurdo.
\end{enumerate}
\textbf{Teorema: compattezza per successioni}\\
Sia $(X,d)$ metrico, $E\subset X,E\neq\emptyset$ compatto, allora $\forall\{x_n\}_{n\in\mathds{N}}\subset E$ $\exists\{x_{n_k}\}_{k\in\mathds{N}}$ e $p\in E$ tali che $x_{n_k}\rightarrow p$\\\\
\textbf{Teorema: Heine-Borel}\\
$E\subset\mathds{R}^n$ è compatto $\Leftrightarrow E$ è chiuso e limitato.\\\\
\textbf{Teorema: Bolzano-Weierstrass}\\
Sia $\{x_i\}_{i\in\mathds{N}}$ una successione a valori in $(\mathds{R}^n,d)$ con $d$ metrica euclidea.\\ Se $A=\{x_i;i\in\mathds{N}\}$ è limitato, allora $\exists p\in\mathds{R}^n$ e $\{x_{i_k}\}_{k\in\mathds{N}}$ tali che $x_{i_n}\rightarrow p$\\\\
\textbf{Dimostrazione}\\
Se $A$ è limitato in $\mathds{R}^n$, allora $\exists z\in\mathds{R}^n$ e $r>0$ tali che $A\subset B_r(z)\subset \bar{B_r}(z)$ che è compatto.\\
\\
\textbf{Teorema: Weierstrass}\\
Sia $f:k\in(X,d)\rightarrow\mathds{R}$, con $(X,d)$ metrico e compatto, f continua, allora $\exists x_M,x_m\in k:$ $f(x_M)=\max f(x)$ e $f(x_m)=\min f(x)$\\\\
\textbf{Dimostrazione}\\
Considero solo il massimo, l'altro caso è analogo.
\begin{itemize}
    \item supponiamo per assurdo $\sup k=+\infty$. Allora $\exists \{x_n\}\in k:$ $f(x_n)\rightarrow+\infty$. Per la compattezza di $k$ $\exists \{x_{n_k}\}$ e $x\in k$ tali che $x_{n_k}\rightarrow x$. Per la continuità di f in k $f(x_{n_k})\rightarrow f(x)\in\mathds{R}$, ma $+\infty\notin\mathds{R}$.
    
    \item
\end{itemize}
Una funzione $f:E\subset(X_1,d_1)\rightarrow(X_2,d_2)$ si dice uniformemente continua se $\forall\epsilon>0$ $\exists\delta=\delta(\epsilon):$ $\forall x,p\in E$ con $d_1(x,p)<\delta$ si ha $d_2(f(x),f(p))<\epsilon$. Una funzione uniformemente continua è anche continua.\\\\
\textbf{Teorema: Heine-Cantor}\\
Sia $f:k\subset(X_1,d_1)\rightarrow(X_2,d_2)$ una funzione continua in $k$ compatto, allora $f$ è uniformemente continua.

\section{Calcolo differenziale}
\subsection{La derivata}
Sia $f:(a,b)\in\mathds{R}\rightarrow\mathds{R}$ e $x_0\in(a,b)$, essa è derivabile in $x_0$ se esiste finito il limite $\lim_{h\rightarrow0}\frac{f(x_0+h)-f(x_0)}{h}$. Derivabile in $(a,b)$ se è derivabile $\forall x\in(a,b)$.\\\\
$\frac{f(x_0+h)-f(x_0)}{h}$ è detto rapporto incrementale e il suo limite è detto derivata della funzione nel punto: $f'(x_0)=\lim_{h\rightarrow0}\frac{f(x_0+h)-f(x_0)}{h}$\\
Il rapporto incrementale corrisponde al coefficiente angolare della retta passante per i punto $p=(x_0,f(x_0)$ e $q=(x_0+h,f(x_0)+h$. In una funzione derivabile il rapporto al limite raggiunge il coefficiente angolare della tangente nel punto $x_0$.\\\\
La derivata può anche essere definita come $f'(x_0)=\lim_{x\rightarrow x_0}\frac{f(x)-f(x_0)}{x-x_0}$ che equivale a porre $x=x_0+h$.\\
La condizione di derivabilità può essere riscritta come: $\exists f'(x_0)\in\mathds{R}:$ $f(x)=f(x_0)+f'(x_0)(x-x_0)+o(x-x_0)$ per $x\rightarrow x_0$\\\\
\textbf{Teorema: derivabilità $\rightarrow$ continuità}\\
Sia $f:(a,b)\rightarrow\mathds{R}$ una funzione derivabile in $x_0\in(a,b)$, f è continua in $x_0$\\\\
\textbf{Dimostrazione}\\
Dalla derivabilità di $f$ in $x_0$: $f(x)=f(x_0)+f'(x_0)(x-x_0)+o(x-x_0)$ per $x\rightarrow x_0$, quindi $\lim_{x\rightarrow x_0}f(x)=f(x_0)$

\subsection{Punti di non derivabilità}
Sia $f:(a,b)\rightarrow\mathds{R}$ una funzione continua in $(a,b)$, essa non è necessariamente derivabile sull'intervallo. I punti in cui non è derivabile sono detti punti di non derivabilità.\\\\
Alcuni punti di non derivabilità notevoli sono:\begin{itemize}
    \item \textbf{Punto angoloso:} una funzione ha un punto angoloso in $x_0$ se esistono finiti i limiti del rapporto incrementale da destra e da sinistra per $x\rightarrow x_0$, ma sono diversi. (es: $f(x)=|x|$)
    \item \textbf{Cuspide:} una funzione presenta una cuspide in $x_0$ se i limiti del rapporto incrementale esistono infiniti e discordi. (es:$f(x)=\sqrt{|x|}$)
    \item \textbf{Flesso a tangente verticale:} la tangente alla funzione in un punto è verticale se i limiti del rapporto incrementale esistono infiniti e uguali. (es:$f(x)=\sqrt[3]{x}$)
\end{itemize}

\subsection{Teoremi fondamentali del calcolo differenziale}
La derivata di una funzione può fornire informazioni importanti sul suo andamento attraverso ad acluni teoremi.\\\\
Sia $f:I\in\mathds{R}\rightarrow\mathds{R}$ una funzione derivabile in $I$ intervallo, un punto $x_0\in I$ è detto massimo relativo (o locale) se $\exists\delta>0:$ $f(x)\leq f(x_0)$ $\forall x\in I\cap B_\delta(x_0)$. Una definizione analoga vale per il minimo relativo.\\\\
\textbf{Teorema: Fermat}\\
Sia $f:(a,b)\rightarrow\mathds{R}$ derivabile in $(a,b)$ e sia $x_0\in(a,b)$ un punto di massimo o minimo relativo per $f$, allora $f'(x_0)=0$\\\\
\textbf{Dimostrazione}\\
Considero il caso di un punto di massimo, l'altro è analogo.\\
$x_0$ massimo $\rightarrow$ $\exists\delta>0:$ $f(x)\leq f(x_0)$ $\forall x\in I\cap B_\delta(x_0)$, quindi $\lim_{x\rightarrow x_0}\frac{f(x)-f(x_0)}{x-x_0}$ risulta $\leq0\forall x\in[x_0,x_0+\delta)$ e $\geq0\forall x\in(x_0-\delta,x_0]$, ma $f$ è derivabile in $x_0$, quindi $f'(x_0)=0$\\\\
La conseguenza di questo teorema è che dato un intervallo i punti di massimo e minimo vanno cercati agli estremi, nei punti di non derivabilità e nei punti con derivata nulla.\\\\
\textbf{Teorema: Rolle}\\
Sia $f:[a,b]\rightarrow\mathds{R}$ una funzione continua in $[a,b]$, derivabile in $(a,b)$ e tale che $f(a)=f(b)$, allora $\exists  x_0\in(a,b):$ $f'(x_0)=0$\\\\
\textbf{Dimostrazione}\\
Se la funzione è costante $f'(x)=0\forall x\in [a,b]$, altrimenti per il teorema di Weierestrass $f(x)$ ha massimo e minimo e per il teorema di Fermat la derivata in questi punti è nulla.\\\\
\textbf{Teorema: valor medio o di Lagrange}\\
Sia $f:[a,b]\rightarrow\mathds{R}$ una funzione continua in $[a,b]$ e derivabile in $(a,b)$, allora $\exists x_0\in(a,b):$ $\frac{f(b)-f(a)}{b-a}=f'(x_0)$\\\\
\textbf{Dimostrazione}\\
Sia $g(x)=f(x)-[f(a)+\frac{f(b)-f(a)}{b-a}(x-a)]$ $\forall x\in[a,b]$. $g(x)$ verifica le ipotesi del teorema di Rolle, quindi $\exists x_0\in(a,b):$ $g'(x)=0$, ma $g'(x)=f'(x)-\frac{f(b)-f(a)}{b-a}\Rightarrow f'(x_0)=\frac{f(a)-f(b)}{b-a}$\\\\
\textbf{Teorema: monotonia e derivata}\\
Sia $f:I\subset\mathds{R}\rightarrow\mathds{R}$ una funzione derivabile in $I$ intervallo, allora:
\begin{itemize}
    \item $f$ crescente $\Leftrightarrow f'(x)\geq0$ $\forall x\in I$
    \item $f$ decrescente $\Leftrightarrow f'(x)\leq0$ $\forall x\in I$
\end{itemize}
\textbf{Dimostrazione}\\
Dimostriamo il primo caso, l'altro è analogo.
\begin{itemize}
    \item Supponiamo per assurdo $f$ crescente, ma $\exists x_o\in I:$ $f'(x_0)<0$.\\
    $f(x)=f(x_0)+f'(x_0)(x-x_0)+o(x-x_0)$ per $x\rightarrow x_0\Rightarrow$ $f(x)-f(x_0)=(x-x_0)[f'(x_9+o(1)]$. Per $x>x_0$ $f(x)-f(x_0)>0$, $(x-x_0)>0$, ma $f'(x_0)+o(1)<0$ che è assurdo.
    \item $\forall x_1,x_2\in I$ con $x_2>x_1$ $f:(x_1,x_2)\rightarrow\mathds{R}$ verifica le ipotesi del teorema di Lagrange, quindi $\frac{f(x_2)-f(x_1)}{x_2-x_0}=f'(x_0)$. Se $f'(x_0)>0$, allora $f(x_2)-f(x_1)=f'(x_0)\cdot(x_2-x_1)\geq0$, quindi $f(x_2)>f(x_1)$
\end{itemize}

\newpage
\textbf{ToDo:}
\begin{enumerate}
    \item No $\bar{9}$ in $\mathds{Q}$ (algoritmo e dim con serie?)
    \item Completezza
    \item Tabella limiti notevoli
    \item Tabella sviluppi in serie
    \item Dim criterio rapporto per serie
    \item Dim Leibniz
    \item Dim lim successioni $\rightarrow$ funzioni
    \item Dim teorema degli 0 e intervalli inscatolati
    \item Definizione massimi e minimi globali?
    \item Dim Weierstrass
    \item Tabella derivate e regole derivazione
\end{enumerate}


\newpage
\renewcommand*\contentsname{Indice}
\tableofcontents
\end{document}
