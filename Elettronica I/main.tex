\documentclass{article}
\usepackage{graphicx} % Required for inserting images

% importa moduli
\usepackage{graphicx} % Required for inserting images
\usepackage{dsfont}
\usepackage{amssymb}
%\usepackage{cases}
\usepackage{ifthen}
\usepackage{geometry}
 \geometry{
 a4paper,
 total={180mm,280mm},
 left=15mm,
 top=5mm,
 }
\usepackage{stackengine}
\usepackage{amsmath} 
\usepackage{mathtools}
\newcommand\ubar[1]{\stackunder[1.2pt]{$#1$}{\rule{.8ex}{.075ex}}}
\usepackage{graphicx}
\usepackage{wrapfig}


\title{Elettronica I}
\author{Luca Vettore}
\date{Ottobre 2023}

\begin{document}

% importa moduli
\usepackage{ dsfont }
\usepackage{ amssymb }
\usepackage{cases}
\usepackage{ifthen}
\usepackage{geometry}
 \geometry{
 a4paper,
 total={180mm,280mm},
 left=15mm,
 top=5mm,
 }
\usepackage{stackengine}
\usepackage{amsmath} 
\usepackage{mathtools}
\newcommand\ubar[1]{\stackunder[1.2pt]{$#1$}{\rule{.8ex}{.075ex}}}
\usepackage{graphicx}
\usepackage{wrapfig}
%\graphicspath{ {./ims/} }

% importa immagini
\newcommand{\im}[2]{
\begin{center}
\includegraphics[width=#1\textwidth]{#2}
\end{center}
}

\newcommand{\ims}[3]{
\begin{center}
\includegraphics[width=#1\textwidth]{#2}
\includegraphics[width=#1\textwidth]{#3}
\end{center}
}

\newcommand{\imsd}[4]{
\begin{center}
\includegraphics[width=#1\textwidth]{#3}
\includegraphics[width=#2\textwidth]{#4}
\end{center}
}

% insiemi numerici
\newcommand{\R}{\mathds{R}}
\newcommand{\N}{\mathds{N}}
\newcommand{\C}{\mathds{C}}

% derivate parziali 
\newcommand{\pd}[2]{\frac{\partial #1}{\partial #2}}

% indice
\newcommand{\Index}{
\newpage
\renewcommand*\contentsname{Indice}
\tableofcontents
}

% teoremi
\newcommand{\teo}[3]{
\textbf{Teorema #1}\\
#2\\
\ifthenelse{\equal{#3}{}}{}{\textbf{Dimostrazione}\\
#3}
}

% matrici
\newcommand{\mat}[1]{{
  \tiny\arraycolsep=0.3\arraycolsep\ensuremath{\begin{pmatrix}#1\end{pmatrix}}}}
  
% nabla
\newcommand{\Nabla}[]{\ubar{\nabla}}

% cambia altezza righe tabelle
\renewcommand{\arraystretch}{1.5}



\maketitle

\section{Introduzione}

\subsection{Che cosa è l'elettronica}
L’elettronica è la disciplina scientifico-tecnologica che si occupa della
generazione, del trasporto, del controllo e della raccolta di particelle
subatomiche dotate di massa e di carica elettrica (come, ad esempio, gli
elettroni).\\
Gli elettroni sono adatti ad essere impiegati nei sistemi per l’elaborazione, la
trasmissione e l’archiviazione delle informazioni.\\
È possibile far muovere un elevato numero di elettroni a grande velocità e
impiegando un ridotta quantità di energia, perché la massa delle particelle è
piccola.\\
L’elettronica permette di realizzare sistemi di acquisizione, elaborazione, controllo
e memorizzazione molto complessi, efficienti e veloci.

\subsection{Unità di misura}
Le unità di misura fondamentali del sistema internazionale sono le seguenti:

\begin{center}
\begin{tabular}{|c|c|p{6cm}|}
\hline
\textbf{Unità di misura} & \textbf{Simbolo} & \textbf{Grandezza misurata} \\
\hline
Metro & m & Lunghezza o distanza \\
Chilogrammo & kg & Massa \\
Secondo & s & Tempo \\
Ampere & A & Corrente elettrica \\
Kelvin & K & Temperatura \\
Mole & mol & Quantità di sostanza \\
Candela & cd & Intensità luminosa \\
\hline
\end{tabular}
\end{center}
Combinando le unità fondamentali, è possibile definire delle unità derivate. In elettronica sono utili le seguenti:

\begin{center}
\begin{tabular}{|c|c|p{6cm}|c|}
\hline
\textbf{Unità di misura} & \textbf{Simbolo} & \textbf{Grandezza misurata} & \textbf{Definizione}\\
\hline
Volt & V & Tensione elettrica & $J/(A\cdot s)$\\
Ohm & $\Omega$ & Resistenza elettrica & $V/A$\\
Joule & J & Energia & $N\cdot m$\\
Watt & W & Potenza elettrica & $J\cdot s$\\
Farad & F & Capacità elettrica & $C/V$\\
Hertz & Hz & Frequenza & $1/s$ \\
Henry & H & Induttanza & $(V\cdot s)/A$\\
Coulomb & C & Carica elettrica & $A\cdot s$\\
Tesla & T & Induzione magnetica & $(V\cdot s)/m^2$\\
\hline
\end{tabular}
\end{center}
Le grandezze misurate possono appartenere a ordini di grandezza molto diversi tra loro. Per questo motivo si introducono dei prefissi moltiplicativi da porre davanti alle unità di misura per specificarne il numero di zeri da aggiungere.\\\\
Di seguito alcuni dei prefissi più comuni e l'ordine di grandezza associato, oltre all'abbreviazione da usare per i simulatori circuitali basati su SPICE:

\begin{center}
\begin{tabular}{|c|c|c|c|}
\hline
\textbf{Nome} & \textbf{Simbolo} & \textbf{Valore} & \textbf{Abbreviazione SPICE} \\
\hline
Yotta & Y & $10^{24}$ & \\
Zetta & Z & $10^{21}$ & \\
Exa & E & $10^{18}$ & \\
Peta & P & $10^{15}$ & \\
Tera & T & $10^{12}$ & T \\
Giga & G & $10^9$ & G \\
Mega & M & $10^6$ & MEG \\
Kilo & k & $10^3$ & K\\
Hetto & h & $10^2$ & \\
Deca & da & $10^1$ & \\
Deci & d & $10^{-1}$ & \\
Centi & c & $10^{-2}$ & \\
Milli & m & $10^{-3}$ & M \\
Micro & $\mu$ & $10^{-6}$ & U \\
Nano & n & $10^{-9}$ & N \\
Pico & p & $10^{-12}$ & P \\
Femto & f & $10^{-15}$ & F \\
Atto & a & $10^{-18}$ & \\
Zepto & z & $10^{-21}$ & \\
Yocto & y & $10^{-24}$ & \\
\hline
\end{tabular}
\\(la maggior parte dei simulatori SPICE non differenzia tra maiuscole e minuscole)
\end{center}

\subsection{Grandezze}
In questa sezione sono presentate alcune delle grandezze più importanti nello studio dell'elettronica.

\subsubsection{Carica elettrica}
La carica elettrica è una proprietà fondamentale della materia e si misura in coulomb ($1C=1A\cdot s$). Questa grandezza e le sue proprietà stanno alla base del funzionamento di tutti i sistemi elettronici.\\\\
La carica elettrica è quantizzata: le cariche sono sempre multipli (positivi o negativi) della carica elementare $q_e=1.6021\cdot 10^{-19}C$.\\\\
Normalmente la materia è elettricamente neutra, ma è possibile separare cariche negative e positive in alcuni processi. Nei solidi esistono cariche fisse e mobili. Le cariche mobili sono quelle che danno origine alle correnti, mentre le cariche fisse possono avere una grande importanza nel funzionamento di alcuni dispositivi (ad es quelli a semiconduttori).

\subsubsection{Corrente elettrica}
Uno spostamento di cariche all'interno di un conduttore è chiamato corrente elettrica. L'intensità di corrente è definita come
$$ I=\frac{dQ}{dt} $$
e si misura in ampere ($1A=1C/s$).\\\\
Nonostante il concetto di carica risulti più fondamentale rispetto alla corrente, si è deciso di inserire l'ampere tra le unità di misura fondamentali perché è decisamente più facile generare una corrente piuttosto che isolare una carica elettrica.

\subsubsection{Differenza di potenziale}
Due particelle dotate di carica interagiscono tra loro con una forza di intensità proporzionale al prodotto delle cariche. Questa forza è chiamata forza di Coulomb ed è conservativa.\\
Possiamo definire il campo elettrico come rapporto tra la forza subita da una particella di carica q in un punto dello spazio e il valore della carica stessa:
$$ E=\frac{F_c}{q} $$
Questa grandezza si misura in newton/coulomb o equivalentemente in volt/metro.\\\\
A questo campo possiamo associare un potenziale. La differenza di potenziale tra due punti dello spazio è chiamata tensione o voltaggio e si misura in volt ($1V= \frac{1Kg\cdot m^2}{A s^3}$).








\newpage
\section{Nodi e maglie di un circuito}
Un punto dove si incontrano più terminali di elementi circuitali è detto nodo.
\im{0.5}{ims/nodi}
In base al modo in cui è disegnato un circuito i nodi potrebbero non risultare puntiformi. Le due immagini rappresentano lo stesso circuito, con lo stesso numero di nodi, ma disegnati diversamente. La rappresentazione di sinistra può risultare più ordinata e comprensibile, mentre ridurre il numero di collegamenti come nel caso di destra aiuta nella risoluzione dei circuiti.\\\\
Una maglia è un percorso chiuso attraverso due o più elementi circuitali.
\im{0.35}{ims/maglie}
Nel conteggio delle maglie bisogna stare attenti a considerare sia le maglie elementari (che non circondando altre maglie), che eventuali maglie esterne. Questa distinzione ha senso solo nel caso di circuiti planari, cioè che possono essere rappresentati su un piano senza incrociare le connessioni.

\subsection{Leggi di Kirchoff}
Le leggi di Kirchoff permettono di calcolare le grandezze elettriche all'interno di un circuito in corrente continua usando i concetti di nodi e maglie.\\\\
La prima legge di Kirchoff ("KVL" o "Kirchoff voltage law") prevede che la somma (con segno) delle tensioni lungo qualsiasi maglia del circuito debba essere nulla:
$$ \sum_{k\in\text{maglia}} V_k=0 $$
Questa legge è conseguenza della conservatività del campo elettrico ($\Rightarrow\oint_\gamma E=0\quad \forall \gamma$).\\\\
La seconda legge di Kirchoff ("KCL" o  "Kirchoff circuit law") prevede che la somma (con segno) delle correnti in un nodo debba essere nulla:
$$ \sum_{k\in\text{nodo}} I_k=0 $$
Questa legge è conseguenza della legge di conservazione della carica, con l'ipotesi aggiuntiva che non vi sia accumulo di carica in un nodo.\\\\
Entrambe le leggi rimangono valide indipendentemente dall'ordine in cui sono considerati i componenti e dal segno dato alle grandezze elettriche. Nonostante ciò, nella risoluzione di un circuito è utile scegliere una convenzione sui segni e mantenerla fino alla fine, per evitare contraddizioni ed errori. In questi appunti viene usata la convenzione degli utilizzatori.

\subsubsection{KCL nel dominio del tempo}
Per un nodo a cui non sono collegati generatori di tensione, la legge di Kirchoff per la corrente può essere scritta come:
$$ i(v)+\frac{d}{dt}q(v)+u=0 $$
dove la dipendenza da t è sottintesa.\\
Le grandezze utilizzate rappresentano:
\begin{itemize}
    \item $i(v)$: la corrente nei bipoli resistivi
    \item $\frac{d}{dt}q(v)$: corrente nei bipoli capacitivi
    \item $u(t)$: segnali in ingresso prodotti da generatori di corrente
\end{itemize}

\subsection{Teorema di Tellegen}
In qualsiasi circuito, la somma (con segno) della potenza di tutti i bipoli è nulla.\\
La potenza può essere espressa come $P=VI$, quindi:
$$ \sum_kP_k=\sum_kV_kI_k=0 $$

















\newpage
\section{Grandezze elettriche variabili nel tempo}
\subsection{Nel dominio del tempo}
\subsubsection{Prime definizioni}
Una grandezza elettrica variabile nel tempo è detta segnale. I segnali possono essere di tensione ($v(t)$) o di corrente ($i(t)$), in base a quale delle grandezze viene fatta variare. La dipendenza dal tempo è spesso omessa per alleggerire la notazione, ma i segnali vengono di solito scritti con la lettera minuscola per distinguerli dalle grandezze costanti.\\\\
Una distinzione utile da fare è quella tra segnali continui (solitamente usati nei circuiti analogici) e campionati (solitamente usati da sistemi digitali):
\begin{itemize}
    \item \textit{Continuo:} il valore può cambiare in qualsiasi momento
    \item \textit{Campionato:} il valore cambia in corrispondenza con un segnale detto di clock o temporizzazione.
\end{itemize}
\im{0.8}{ims/segnali}
Una grandezza che può assumere un numero infinito di valori è detto analogico, mentre se il suo valore cambia in modo discreto è detto digitale.

\subsubsection{Segnali periodici}
Un segnale è detto periodico se si ripete dopo un tempo $T$ detto periodo:
$$ x(t+T)=t \forall t $$
L'inverso del periodo è la frequenza: $f=\frac{1}{T}$\\
Si definisce la frequenza angolare come: $\omega=2\pi f$\\\\
Il valore medio di un segnale periodico è dato da:
$$ v_m=\frac{1}{T}\int_0^Tx(t)dt $$
che spesso vale 0. Una grandezza più utile si ottiene calcolando il valore di una grandezza elettrica costante che trasferisce la stessa quantità di potenza. Questa grandezza è chiamata valore efficace (o "root-mean-square") ed è data da:
$$ v_{rms} = \sqrt{\frac{1}{T}\int_0^T(x(t))^2dt}=|x(t)|_2$$
Un'altra grandezza utile per i segnali periodici è l'ampiezza (picco picco o peak to peak), che corrisponde alla distanza tra massimo e minimo del segnale.

\subsubsection{Guadagno}
Si definisce guadagno in potenza di un circuito il rapporto tra la potenza erogata dal generatore di segnale collegato a una resistenza di carico con (b) e senza (a) il circuito (amplificatore):
$$ G = \frac{P_{a}}{P_{b}} $$
Solitamente il guadagno è espresso in decibel (decimi di bel), che, usando una scala logaritmica, permettono di lavorare con valori che assumono ordini di grandezza molto diversi.
$G_{dB}=10\log(G)=10\log(\frac{P_a}{P_b})$
Ricordando che vale $P_i=\frac{V^2_i}{R}$, possiamo esprimere il guadagno in potenza in funzione della tensione:
$$ G = \frac{RV^2_b}{RV_a^2}=\left( \frac{V_b}{V_a} \right)^2 $$
$$ G_{dB} = 10\log(G)=10\log\left(\frac{V_b}{V_a}\right)^2=20\log\left(\frac{V_b}{V_a}\right)  $$
La costante davanti al logaritmo diventa 20 se si esprime il guadagno in termini di tensione (o corrente, seguendo un procedimento analogo).\\\\
Collegando l'uscita di un circuito all'entrata di un alto si compie un collegamento detto in cascata. Il guadagno di due circuiti in cascata si può calcolare come (omessi i semplici passaggi algebrici):
$$ G_{tot} = G_1\cdot G_2 $$
$$ G_{dB}^{tot} = G^1_{dB}+G^2_{dB} $$

\subsubsection{Risposta impulsiva}
La risposta impulsiva $h(t)$ di un circuito (lineare e invariante nel tempo) è la sua risposta ad un segnale formato da una delta di dirac. Questa grandezza permette di studiare la risposta di un filtro ad un segnale generico $x_i(t)$ attraverso all'operazione di convolluzione:
$$ x_o(t) = x_i(t)*h(t) $$
Calcolare la risposta di un circuito in questo modo richiede la risoluzione di un integrale e può risultare poco pratico. Vedremo nella sezione successiva come le proprietà della trasformata di fourier permettano di semplificare il conto, passando alla risposta in frequenza.

\subsection{Analisi nel dominio della frequenza}
Spesso l'analisi di segnali nel dominio del tempo può risultare laboriosa (nel dominio del tempo le relazioni tensione-corrente di condensatori e induttanze sono euqazioni differenziali). Esistono alcuni strumenti che permettono di semplificare i problemi e i conti nel caso di circuiti complicati. Uno di questi strumenti è l'analisi di Fourier, che permette di studiare un segnale nel dominio della frequenza (dove ad esempio le relazioni tensione-corrente di induttanze e condensatori sono funzioni razionali).

\subsubsection{Serie di Fourier}
Un segnale periodico può essere scomposto in armoniche (multipli interi della frequenza fondamentale) ed espresso come serie di Fuorier (sotto ipotesi abbastanza deboli e facilmente verificate per segnali elettrici realistici):
$$ x(t) = \frac{1}{2}a_0+\sum_{k=1}^{+\infty}(a_k\cos(2k\pi f_0t)+ b_k\sin(2k\pi f_0t)) $$
I coefficienti della serie si possono trovare con i seguenti integrali:
$$ a_k = \frac{T}{2}\int_{-\frac{T}{2}}^{\frac{T}{2}}x(t)\cos(2k\pi f_0t)dt $$
$$ b_k = \frac{T}{2}\int_{-\frac{T}{2}}^{\frac{T}{2}}x(t)\sin(2k\pi f_0t)dt $$
Lo stesso segnale può essere scomposto su una base di esponenziali complessi (usando l'identità di eulero), ottenendo una forma più compatta della serie di Fourier:
$$ x(t) = \sum_{K=-\infty}^{\infty}c_ke^{j2k\pi f_0t} $$
dove abbiamo usato $j$ per indicare l'unità immaginaria, per distinguerla dal simbolo usato per la corrente. I coefficienti di Fourier in questa base sono dati da:
$$ c_k=\bar{c}_{-k}=\frac{a_k-jb_k}{2}=\frac{1}{T}\int_{\frac{T}{2}}^{\frac{T}{2}}x(t)e^{-j2k\pi f_0t}dt $$

\subsubsection{Trasformata di Fourier}
Possiamo estendere i risultati precedenti a segnali non periodici, considerandoli segnali periodici nel limite $T\rightarrow+\infty$ e $f_0\rightarrow0$. Possiamo quindi scrivere:
$$ x(t) = \int_{+\infty}^{-\infty}X(f)e^{j2\pi ft}df $$
La funzione $X(f)$ è detta trasformata di Fourier di $x(t)$. Invertendo la relazione precedente:
$$ X(f)=\int_{+\infty}^{-\infty}x(t)e^{-j2\pi ft}dt $$
L'operatore che associa a una funzione la sua trasformata di Fourier è di solito rappresentato con $\mathcal{F} $ e il suo inverso (antitrasformata di Fourier) con $\mathcal{F}^{-1}$.\\\\
Per funzioni continue è possibile passare da funzione a trasformata e viceversa attraverso ai due operatori:
$$ \mathcal{F}^{-1}\mathcal{F}x(t) = x(t) $$
mentre per funzioni discontinue si avranno problemi nei punti di discontinuità (solitamente l'inversa assumerà un valore intermedio tra i limiti destro e sinistro).\\\\
Le proprietà fondamentali della trasformata di Fourier sono le seguenti:
\begin{itemize}
    \item Linearità: 
    $$\mathcal{F}\{a x(t) + b g(t)\} = a \mathcal{F}\{x(t)\} + b \mathcal{F}\{g(t)\}$$
    con $a,b$ costanti.
    
    \item Traslazione nel tempo:
    $$ \mathcal{F}\{f(t - t_0)\} = e^{-j 2 \pi f t_0} \mathcal{F}\{x(t)\} $$

    \item Traslazione in frequenza:
    $$ \mathcal{F}\{e^{j 2 \pi f_0 t} x(t)\} = F(f - f_0) $$

    \item Cambio di scala:
    $$ \mathcal{F}\{f(at)\} = \frac{1}{|a|} F\left(\frac{f}{a}\right) $$
    con $a\neq0$ costante.

    \item Moltiplicazione:
    $$ \mathcal{F}\{x(t) \cdot g(t)\} = \mathcal{F}\{x(t)\} * \mathcal{F}\{g(t)\} $$
    dove $*$ indica l'operazione di convolluzione.

    \item Convolluzione:
    $$ \mathcal{F}\{x(t) * g(t)\} = \mathcal{F}\{x(t)\} \cdot \mathcal{F}\{g(t)\} $$

    \item Derivazione:
    $$ \mathcal{F}\{f'(t)\} = (j 2 \pi f) X(f) $$

    \item Integrazione:
    $$ \mathcal{F}\left\{\int x(t)\right\} = \frac{1}{i 2 \pi t}X(f) $$

    \item Parità:\\
    $x(t)$ reale pari $\leftrightarrow$ $X(f)$ reale pari\\
    $x(t)$ reale dispari $\leftrightarrow$ $X(f)$ immaginaria dispari

    \item Periodicità:\\
    $x(t)$ periodico $\leftrightarrow$ $X(f)$ campionato\\
    $x(t)$ campionato $\leftrightarrow$ $X(f)$ periodico  

\end{itemize}

\subsubsection{Risposta in frequenza}
La trasformata di Fourier della risposta impulsiva di un circuito è detta risposta in frequenza $H(f)$ e permette di calcolare facilmente la risposta di un circuito lineare ad un impulso generico (nel dominio della frequenza). Svolgendo la trasformata di Fourier della relazione trovata per la risposta impulsiva otteniamo:
$$ X_o(f) = X_i(f)\cdot H(f) $$
L'operazione di convolluzione è quindi stata sostituita da un prodotto.\\\\
La risposta in frequenza in generale è un numero complesso, possiamo quindi rappresentarla in coordinate polari. In tal caso:
$$ H(f) = |H(f)|e^{j\theta(H(f))} $$
Applicandola ad un segnale generico si nota che il modulo agisce come una contrazione, mentre la fase agisce come uno sfasamento.\\\\
La trasformata di Fourier di un segnale sinusoidale con frequenza $f$ è una coppia di delta di Dirac, alle frequenze $\pm f$. L'uscita di un circuito lineare a cui è applicata una sinusoide in ingresso, è anch'esso una sinusoide alla stessa frequenza. Ad un circuito lineare possiamo inolte applicare il principio di sovrapposizione degli effetti.
Queste osservazioni, insieme al formalismo di Fourier, ci permettono di vedere qualsiasi segnale come sovrapposizione di sinusoidi a diverse frequenze, ognuna corrispondente a una coppia di $\delta$, e la risposta in frequenza come una funzione che fornisce il coefficiente (complesso) per cui moltiplicare ognuna di queste per ottenere il segnale in uscita.\\\\
Il guadagno del circuito può essere calcolato come:
$$ G_{dB}(f)=20\log H $$

\subsubsection{Impedenza}
Applicando la trasformata di Fourier alle relazioni tensione-corrente (e sfruttando le sue proprietà), possiamo ottenere una relazione del tipo:
$$ V(f) = Z(f)I(f) $$
che vale anche per condensatori e induttori.\\\\
La grandezza $Z(f)$ è un numero complesso ed è detta impedenza. Può essere separata in parte reale e parte immaginaria:
$$ Z=R+jX $$
dove $R$ è la resistenza ($\geq0$) e $X$ è detta reattanza (può avere qualsiasi segno).\\\\
L'impedenza equivalente di un circuito con bipoli in serie o in parallelo può essere calcolata usando le stesse regole viste per la resistenza.\\\\
L'inverso dell'impedenza è detto ammettenza:
$$Y=\frac{1}{Z}=G+jB $$
dove $G$ è la conduttanza e $B$ è detta suscettanza.

\subsubsection{Diagrammi di Bode e Nyquist}
(...)












\newpage
\section{Bipoli, n-poli}
Un circuito è formato da elementi (o componenti) circuitali connessi tra loro. Gli elementi più semplici sono i bipoli elettrici.\\\\
\im{0.3}{ims/Bipolo.png}
I bipoli sono caratterizzati da due terminali, il polo positivo (+) e quello negativo (-). Usando la convenzione degli utilizzatori (non necessaria, ma utile per rimanere coerenti con le definizioni durante la risoluzione di circuiti):
\begin{itemize}
    \item la tensione è misurata dal polo negativo a quello positivo
    \item la corrente è considerata positiva se entra nel bipolo e negativa se ne esce
\end{itemize}
Un bipolo può essere descritto attraverso a una relazione tra tensione e corrente, nota come caratteristica tensione-corrente. Questa relazione può essere espressa in diversi modi:
\begin{itemize}
    \item forma implicita: $f(V,I)=0$ (questa forma esiste sempre)
    \item forma esplicita rispetto a I: $I=g(V)$ (se esiste)
    \item forma esplicita rispetto a V: $V=h(I)$ (se esiste)
\end{itemize}
Se la caratteristica tensione-corrente di un bipolo in forma esplicita rispetto a I o V è una funzione lineare, allora il bipolo è detto lineare. In questo caso la rappresentazione grafica della caratteristica sarà una retta.\\\\
Il comportamento di un bipolo in risposta ad un segnale variabile nel tempo, può essere descritto attraverso alla sua impedenza.\\\\
Un elemento circuitale con più di 2 terminali è detto n-polo o multipolo. Il comportamento di questi elementi è più complicato rispetto ai bipoli e va studiato caso per caso.

\subsection{Resistore}
Il resistore è il bipolo lineare più semplice. La sua caratteristica tensione-corrente è data dalla legge di Ohm:
$$ V-RI=0 $$
La costante R è chiamata resistenza e si misura in ohm ($1\Omega=1\frac{V}{A}$).\\\\
L'inverso di R è detto conduttanza e si misura in siemens ($1S=\Omega^{-1}$) e la legge di ohm può essere riscritta in termini di conduttanza:
$$ I=GV $$
Un resistore dissipa una potenza pari al prodotto tra corrente e tensione:
$$ P=VI $$
\im{0.3}{ims/resistore}
L'impedenza di un resistore coincide con la sua resistenza ($\Leftarrow$ la resistenza è reale, positiva e indipendente da tempo e frequenza):
$$ Z_R=R $$

\subsection{Generatore di tensione/corrente}
Il generatore di tensione è un bipolo che mantiene una tensione costante tra i terminali.\\
La sua caratteristica tensione-corrente è:
$$ V=V_0 $$
\im{0.3}{ims/gen-tensione}
Il simbolo (a) è utilizzato esclusivamente per generatori di tensione costante, mentre il simbolo (b) viene spesso disegnato insieme alla forma della funzione generata.\\\\
Il generatore di corrente è un bipolo attraversato da una corrente costante.\\
La sua caretteristica tensione-corrente è:
$$ I=I_0 $$
\im{0.3}{ims/gen-corrente}

\subsubsection{Circuiti aperti e cortocircuiti}
Un generatore di corrente spento è attraversato da una corrente nulla. Questa configurazione è chiamata circuito aperto e può essere utilizzata per modellare l'assenza di collegamento.\\\\
Un generatore di tensione spento ha ai capi una tensione nulla. Questa configurazione è chiamata cortocircuito e si usa per rappresentare una connessione con un filo conduttore ($R\rightarrow0$).

\subsubsection{Generatori dipendenti}
I generatori dipendenti sono 4-poli (o doppi bipoli) che generano una grandezza elettrica in funzione di un'altra grandezza elettrica. Questi componenti hanno 4 terminali: 2 in ingresso per la grandezza di controllo e 2 in uscita per la grandezza generata. I generatori dipendenti possono essere di 4 tipi:
\begin{itemize}
    \item \textit{Generatore di tensione controllato in tensione}: \\
    non assorbe corrente in ingresso e genera un segnale dato da: $$ V_o=EV_i $$ dove $E$ è detto guadagno di tensione.\\
    \im{0.3}{ims/VCVS}

    \item \textit{Generatore di corrente controllato in corrente}:\\
    non c'è caduta di tensione tra i terminali in ingresso. Il segnale in uscita è dato da:
    $$ I_o=FI_i $$
    dove $F$ è detto guadagno di corrente.
    \im{0.3}{ims/CCCS}

    \item \textit{Generatore di corrente controllato in tensione}:\\
    non assorbe corrente in ingresso e produce un segnale dato da:
    $$ I_o=GV_i $$
    dove $G$ è detta transconduttanza e ha l'unità di misura di una conduttanza.
    \im{0.3}{ims/VCCS}

    \item \textit{Generatore di tensione controllato in corrente}:\\
    non c'è caduta di tensione tra i terminali d'ingresso e produce un segnale dato da:
    $$ V_o=HI_i $$
    dove $H$ è detta transresistenza e ha le dimensioni di una resistenza.
    \im{0.3}{ims/CCVS}
    
\end{itemize}


\subsection{Amplificatore operazionale}
L'amplificatore operazionale è un generatore di tensione controllato in tensione che amplifica la differenza di potenziale tra i terminali in ingresso:
$$ V_{out}=E(V^+-V^-) $$
\im{0.4}{ims/ampop}
La maggior parte degli amplificatori operazionali sul mercato hanno un unico terminale di uscita, la tensione $V_{out}$ è quindi da intendersi come riferita al nodo di terra.\\
Oltre al terminale in uscita, l'amp-op presenta due terminali in ingresso, indicati rispettivamente da $+$ (ingresso non invertente) e $-$ (ingresso invertente). Gli amp-op reali hanno inoltre due terminali $V_{cc}^\pm$ per l'alimentazione.\\\\
L'amplificatore operazionale ideale è caratterizzato da:
\begin{itemize}
    \item guadagno $E\rightarrow+\infty$
    \item resistenza in uscita $R_{out}=0$
    \item resistenza in ingresso $R_{in}\rightarrow+\infty$
    \item banda passante $B\rightarrow+\infty$
\end{itemize}
Gli amplificatori operazionali sono molto spesso usati in configurazione retroazionata, riportando, cioè, il segnale in uscita a uno (o entrambi) degli ingressi attraverso una rete di componenti passivi (rete di retroazione). La retroazione è detta positiva se il segnale è riportato all'ingresso non invertente, negativa altrimenti. Il comportamento di un amp-op retroazionato può essere studiato attraverso ai grafi di Mason.\\
In generale i due tipi di retroazione hanno comportamenti diversi:
\begin{itemize}
    \item \textit{retroazione negativa}:\\
    $V_{out}$ è stabile, cioè non può divergere spontaneamente. Vale il principio di terra virtuale: i due terminali in ingresso si trovano alla stessa tensione.

    \item \textit{retroazione positiva}:\\
    $V_{out}$ può divergere spontaneamente (in un amp-op reale tende a $V_{cc}^\pm$).
    
\end{itemize}

\subsubsection{Grafi di Mason}
(...)


\subsection{Condensatore}
Il condensatore è costituito da due superfici metalliche separate da un materiale isolante.\\
Un condensatore è in grado di accumulare sulle sue superfici una carica pari a
$$ q(t)=Cv(t) $$
La costante $C$ è detta capacità e si misura in Farad.\\\\
Per un circuito in continua, un condensatore si comporta come un circuito aperto, non lasciando passare corrente tra i suoi terminali.\\
In generale la caratteristica tensione-corrente di un condensatore è:
$$ i=C\frac{dv}{dt} $$
da cui
$$ v=\frac{1}{C}\int i\;dt $$
quindi la tensione è una funzione continua di t.\\\\
Possiamo scrivere potenza istantanea $p$ ed energia immagazzinata $w$ come:
$$ p=vi=Cv\frac{dv}{dt} \quad\Rightarrow\quad w=\frac{1}{2}Cv^2 $$
L'impedenza di un condensatore è inversamente proporzionale alla frequenza (un condensatore è equivalente a un circuito aperto per un segnale in continua) e alla capacità. Vale:
$$ Z_C = \frac{1}{2\pi j C} = -\frac{j}{2\pi C} $$

\subsection{Induttore}
L'induttore è costituito da un filo conduttore avvolto a spirale. All'interno dell'avvolgimento si ha un campo magnetico proporzionale alla corrente:
$$ \Phi=Li $$
una variazione di flusso magnetico provoca una differenza di potenziale ai capi del filo, per cui la relazione tensione-corrente risulta:
$$ v=L\frac{di}{dt} $$
la corrente risulta:
$$ i=\frac{1}{L}\int v\;dt $$
e l'energia interna con un procedimento analogo a quello usato per il condensatore:
$$ w=\frac{1}{2}Li^2 $$
L'impedenza di un induttore è direttamente proporzionale alla frequenza (si comporta come circuito chiuso per un segnale in continua) e all'induttanza. In particolare vale:
$$ Z_L=2\pi j L $$

\subsection{Diodo a giunzione}
\im{0.3}{"ims/diodo"}
Il diodo a giunzione è un componente a semiconduttori, formato da una giunzione tra semiconduttore drogato p (ricco di lacune) e n (ricco di elettroni). Il diodo permette il passaggio di corrente dalla regione p (anodo) alla regione n (catodo), una volta superata la tensione di soglia ($\simeq 0.7V$ per i più comuni diodi al silicio).\\\\
Il diodo ideale ha tensione di soglia nulla e resistenza nulla nel verso $p\rightarrow n$, mentre nel caso di diodi reali è necessario considerare la tensione di soglia (solitamente $0.7V$), una resistenza e una capacità in serie al componente. Un'altra differenza tra diodi ideali e reali si osserva nel comportamento quando vengono polarizzati inversamente. Il diodo ideale in questo caso è equivalente a un circuito aperto, mentre il diodo reale ha una tensione (tensione di breakdown) al di sopra della quale inizia a condurre (effetto Zener e breakdown a valanga).\\\\
Il comportamento di un diodo reale (comunque con resistenza nulla, se la si vuole considerare è sufficiente aggiungerla in serie) in regione di conduzione (polarizzazione diretta) e interdizione (polarizzazione inversa sotto la tensione di breakdown) è descritto dall'equazione:
$$ I_D = I_S\left( e^{\frac{V_D}{\eta V_T}} - 1 \right)$$
dove $I_D,V_D$ sono la corrente e la tensione del diodo, $1\geq\eta\leq2$ è il fattore di idealità e $V_T=\frac{kT}{e}$ è detta tensione termica.\\\\
I circuiti con i diodi possono spesso essere risolti senza usare l'equazione precedente. Il diodo polarizzato direttamente è infatti spesso approssimabile come un circuito chiuso, mentre la polarizzazione inversa è spesso approssimabile a un circuito aperto. Facendo un ipotesi sulla regione di funzionamento del diodo, è possibile risolvere le grandezze elettriche ai suoi terminali e verificare l'ipotesi fatta. In questo modo è spesso possibile risolvere con buona approssimazione i circiuti con diodi in massimo un paio di tentativi, senza dover introdurre un esponenziale nelle equazioni ottenute dalle leggi di Kirchoff. Bisogna comunque assicurarsi che la non idealità del diodo non risulti importante (basta controllare di essere comodamente entro le regioni di conduzione/interdizione).\\\\
Se oltre alla polarizzazione diretta/inversa è applicato anche un segnale di ampiezza piccola, il comportamento del diodo può essere approssimato dalla retta tangente all'esponenziale. In questo caso la resistenza del diodo può essere approssimata come:
$$ r_d \simeq \frac{\eta V_T}{ID} \quad \text{polarizzazione diretta} \quad r_d\rightarrow\infty \quad \text{polarizzazione inversa}$$
Se si vogliono considerare anche le capacità del diodo nelle due regioni, esistono delle equazioni che permettono di calcolarle partendo dalle sue caratteristiche:
$$ \text{[da aggiungere]} $$
Il diodo, per piccoli segnali, è approssimabile come una resistenza in parallelo ad un condensatore.
Il modello presentato sopra vale solo per segnali piccoli, cioè che non si discostano di molto dal punto di funzionamento in continua e che non fanno passare il componente da una regione di funzionamento a un'altra.


\subsubsection{Diodo Zener}
\im{0.25}{"ims/zener"}
I diodi Zener sono diodi a giunzione con tensione di breakdown nota. Un diodo Zener è approssimabile a un circuito aperto fino alla sua tensione di breakdown, come un circuito chiuso al di sopra. La relazione tensione corrente reale è comunque esponenziale, ma per la maggior parti delle applicazioni pratiche non importa.\\\\
I diodi Zener sono spesso usati per ottenere tensioni di riferimento con valori precisi o per limitare la tensione in una sezione del circuito.

\subsection{Transistor bipolare a giunzione}
\im{0.8}{"ims/bjt"}
Un transistor bipolare a giunzione è un componente ai semiconduttori formato da due giunzioni p-n (pnp) o n-p (npn) a breve distanza. Le due regioni drogate allo stesso modo sono chiamate collettore ed emettitore, mentre la regione con drogaggio opposto è detta base. Questo componente permette di controllare le grandezze elettriche tra collettore ed emettitore attraverso un segnale (di corrente) fornito alla base.\\\\
Avendo 2 giunzioni, ognuna può essere polarizzata in un verso o nell'altro e in totale sono presenti 4 regioni di funzionamento:

\begin{center}
\begin{tabular}{|c|c|c|c|}
    \hline
    Polarizzazione E-B & Polarizzazione C-B & Regione di Funzionamento & Funzione\\
    \hline
    Diretta & Diretta & Saturazione  & $\simeq$ circuito chiuso tra E e C\\
    \hline
    Diretta & Inversa & Regione attiva & $\simeq$ amplificatore di corrente tra B e E \\
    \hline
    Inversa & Inversa & Interdizione & $\simeq$ circuito aperto tra E e C\\
    \hline
    Inversa & Diretta & Regione attiva inversa & $\simeq$ regione attiva (con guadagno ridotto)\\
    \hline
\end{tabular}
\end{center}

\subsubsection{Saturazione e interdizione}
Nella regione di saturazione e interdizione il transistor opera come un interruttore. In saturazione l'interruttore è chiuso, mentre in interdizione è aperto. Queste regioni sono utili nell'elaborazione di segnali digitali, perché permettono di uasre un segnale sulla base per controllare il segnale tra i terminali C ed E. Può inoltre essere usato per realizzare porte logiche.

\subsubsection{Regione attiva}
In regione attiva il transistor opera come un amplificatore di corrente, in cui la corrente alla base e all'emettitore sono legate dalla relazione:
$$ I_c = \frac{\alpha}{1-\alpha}I_b = \beta I_b $$
dove $\alpha$ è l'efficienza e $\beta$ il guadagno del transistor.\\\\
Si può notare che vale:
$$ \beta = \frac{\alpha}{1-\alpha} \simeq \frac{1}{1-\alpha} $$
Per la regione attiva inversa valgono relazioni simili:
$$ I_e = \beta_r I_b $$
con $\beta_r << \beta$.

\subsubsection{Modello di Ebers e Moll}
Il modello di Ebers e Moll descrive il comportamento del transistor in tutte le regioni di funzionamento. Per un transistor pnp:
\im{0.8}{"ims/bjt-equiv"}
dove il pedice $_r$ (reverse) si riferisce alle grandezze in regione attiva inversa e $_f$ (forward) diretta.\\\\
I diodi descrivono le due giunzioni, mentre i generatori controllati l'effetto transistor (amplificazione).
Per un transistor npn il verso dei generatori e dei diodi va invertito.\\\\
Un bjt in regione attiva diretta per piccoli segnali può essere modellato come:
\im{0.5}{"ims/bjt-equiv2"}
dove $r_\pi=\frac{V_t}{I_b}$, $C_\pi=\frac{\tau}{r_\pi}$, $C_\mu=C_j$, $r_\mu\rightarrow\infty$. 
\\Per frequenze basse le capacità possono essere trascurate ed è sufficiente considerare la resistenza $r\pi$ e il guadagno $\beta$

\subsubsection{Effetto Early}
(...)

\newpage
\section{Configurazioni circuitali}
La risoluzione di un circuito si riconduce normalmente alla risoluzione di un sistema lineare. Il numero di equazioni in questo sistema cresce, però, molto velocemente. Per ridurre il numero di equazioni da risolvere è possibile fare delle semplificazioni all'interno del circuito quando si incontrano configurazioni note. In questa sezione sono presentate alcune di queste configurazioni.

\subsection{Collegamento in serie}
Due bipoli attraversati dalla stessa corrente sono detti in serie.
\im{0.5}{ims/serie2}
Per questa configurazione valgono:
$$ I_1=I_2 $$
$$ V=V_1+V_2 $$

\subsubsection{Resistori}
Applicando le leggi di Kirchoff è possibile calcolare il valore del resistore $R_e$ con cui è possibile sostituire due resistori $R_1,R_2$, mantenendo invariate le altre grandezze elettriche. Vale la relazione:
$$ R_e=R_1+R_2 $$
che può essere estesa al caso di n resistori in serie:
$$ R_e=\sum_k^nR_k $$

\subsubsection{Generatori}
La tensione ai capi di n generatori di tensione collegati in serie vale:
$$ V=\sum_k^nV_k $$
Nel caso di generatori di corrente, il collegamento è possibile solo se le loro correnti sono uguali. Altrimenti si otterrebbe una configurazione in cui la corrente entrante nel nodo è diversa da quella uscente.

\subsubsection{Impedenze}
L'impedenza equivalente per bipoli collegati in serie si ottiene come somma delle impedenze:
$$ Z_e = \sum_k Z_k $$


\subsection{Collegamento in parallelo}
Due bipoli che hanno la stessa tensione ai capi sono detti in parallelo.
\im{0.45}{ims/parallelo}
In questo caso si ha:
$$ I=I_1+I_2 $$
$$ V_1=V_2 $$

\subsubsection{Resistori}
Applicando le leggi di Kirchoff, possiamo calcolare il resistore equivalente a due resistori collegati in parallelo:
$$ R_e= \frac{1}{\frac{1}{R_1}+\frac{1}{R_2}}=\frac{R_1R_2}{R_1+R_2}$$
che si estende al caso di n resistori come:
$$ R=\left(\sum_k^n\frac{1}{R_k}\right)^{-1}$$

\subsubsection{Generatori}
Due generatori di tensione possono essere collegati in parallelo solamente se hanno la stessa tensione.\\\\
Nel caso di generatori di corrente collegati in parallelo, la corrente totale è data dalla somme delle correnti:
$$ I=I_1+I_2 $$

\subsubsection{Impedenze}
L'impedenza equivalente per bipoli collegati in parallelo si ottiene come inverso della somma degli inversi delle impedenze dei bipoli:
$$ Z_e = \left( \sum_k \frac{1}{Z_k} \right)^{-1} $$

\subsection{Stella e triangolo}
La stella e il triangolo di resistori sono due configurazioni non semplificabili come collegamenti in serie o in parallelo. In alcuni casi può essere più semplice risolvere una configurazione piuttosto che l'altra e per questo esistono relazioni che permettono di passare da stella a triangolo e viceversa.
\im{0.5}{ims/stella-triangolo}

\subsubsection{Stella $\rightarrow$ triangolo}
\im{0.5}{ims/stella-triangolo2}
Se le tre resistenze sono uguali, allora vale:
$$ R_\Delta=3R_Y $$

\subsubsection{Triangolo $\rightarrow$ stella}
\im{0.5}{ims/stella-triangolo3}
Se le tre resistenze sono uguali, allora vale:
$$ R_Y=\frac{1}{3}R_\Delta $$

\subsection{Generatori equivalenti}
Quando non è necessario calcolare le grandezze elettriche corrispondenti a tutti i bipoli, una rete di resistori e generatori può essere considerata equivalente a un singolo generatore e un resistore.

\subsubsection{Generatore di Thévenin}
Possiamo considerare la rete di resistori e generatori equivalente a un generatore di tensione in serie a una resistenza.
\im{0.6}{ims/thevenin}
Il valore $V_{eq}$ è pari alla tensione di circuito aperto, ottenuta risolvendo il circuito, mentre $R_{eq}$ è pari alla resistenza vista tra i terminali spegnendo tutti i generatori indipendenti.\\\\
Per trovare questi valori, è sufficiente considerare spenti tutti i generatori dipendenti, collegare un generatore di corrente tra i terminali A e B, ricavare la tensione $V_{eq}=V_{ab}$ e quindi $R_{eq}=\frac{V_x}{I_x}$.

\subsubsection{Generatore di Norton}
Possiamo considerare la rete di resistori e generatori equivalente a un generatore di corrente in parallelo a una resistenza.
\im{0.6}{ims/norton}
La corrente $I_{eq}$ è quella ottenuta cortocircuitando i terminali A e B e risolvendo il circuito. Il valore della resistenza è lo stesso del circuito di Thévenin.\\\\
Per i due circuiti equivalenti valgono quindi le relazioni:
$$ R_{eq}^{Norton} = R_{eq}^{Thevenin}$$
$$ V_{eq} = R_{eq}I_{eq} $$

\subsubsection{Massimo trasferimento di potenza}
Consideriamo un circuito equivalente a un circuito di Thévenin, a cui è collegata una resistenza (equivalente) di carico $R_L$.
\im{0.4}{ims/maxP}
Il valore di $R_L$ che massimizza l'assorbimento di potenza dal generatore è:
$$ R_L=R_{eq} $$
\im{0.5}{ims/potenza-carico}
Questo risultato è utile nell'ambito della trasmissione dei segnali, in particolare di segnali deboli, poiché permette di massimizzare la potenza trasferita a un ricevitore.

\subsubsection{Principio di sovrapposizione}
In un circuito formato da componenti con caratteristica tensione-corrente lineare, che contiene generatori indipendenti e dipendenti anch'essi lineari, è possibile considerare separatamente l'effetto di ogni generatore e poi calcolarne la somma.\\\\
Sotto queste ipotesi, un circuito può essere risolto come segue:
\begin{itemize}
    \item si spendono tutti i generatori indipendenti tranne uno
    \item si calcolano tensioni e correnti risultanti
    \item si ripetono i passaggi precedenti per tutti i generatori indipendenti
    \item si sommano le grandezze ottenute
\end{itemize}
Per poter applicare questo procedimento è necessario che \textbf{tutti} i componenti siano lineari, le grandezze elettriche da calcolare siano lineari (ad es: la potenza non è una grandezza che dipende linearmente dalle altre) e che siano solamente in generatori indipendenti a venir rimossi.

\subsection{Circuiti con amplificatori operazionali}
Un amplificatore operazionale retroazionato negativamente può essere usato per costruire diversi circuiti (amplificatori, integratori, derivatori, ...). In generale questi tipi di circuiti possono essere divisi in invertenti e non e può risultare utile conoscerne la forma e la soluzione generale.

\subsubsection{Circuiti invertenti}
\im{0.5}{"ims/amp-inv"}
Il generico circuito invertente con amplificatore operazionale ha questa forma e può essere risolto conoscendo i valori delle due impedenze $Z_1,Z_2$.\\\\
Il guadagno per il circuito invertente generico vale:
$$ A = \frac{v_{out}}{v_{in}} = -\frac{Z_2}{Z_1} $$
Inserendo resistenze al posto delle $Z_i$ si ottiene un amplificatore invertente, inserendo resistenze e condensatori si ottengono integratori e derivatori invertenti.

\subsubsection{Circuiti non invertenti}
\im{0.5}{"ims/amp-ninv"}
Il generico circuito non invertente è leggermente più complicato, ma il suo guadagno si ricava sostanzialmente allo stesso modo:
$$ A = \frac{v_{out}}{v_{in}} = +\frac{Z_2}{Z_1} $$
Anche in questo caso inserendo resistori si ottengono amplificatori (a guadagno finito) e inserendo resistenze e condensatori si ottengono derivatori e integratori.

\subsection{Filtri RC}
L'impedenza di un condensatore varia in modo inversamente proporzionale (in modulo) alla frequenza. Questa proprietà può essere sfruttata per costruire circuiti con una frequenza di taglio (alta o bassa), cioè che permettono il passaggio di segnali al di sotto a al di sopra di una certa frequenza, attenuando (al limite a 0) gli altri. Gli esempi più semplici di questo tipo di circuito sono il circuito RC passa-basso e passa-alto.

\subsubsection{RC passa-basso}
\im{0.35}{"ims/RC1"}
La risposta in frequenza di questo circuito è:
$$H(f) = \frac{1}{1 + j2\pi f RC}$$
da cui si nota che per frequenze basse $H_{db}\rightarrow0$ e $\angle H\rightarrow0$, mentre per frequenze alte $H_{db}\rightarrow-\infty$ e $\angle H \rightarrow-\frac{\pi}{2}$.\\\\
Il circuito lascia quindi (circa) inalterati segnali a basse frequenze e attenua segnali a frequenza alta.\\\\
I diagrammi di bode del circuito sono:
\im{0.8}{"ims/RC2"}
($\tau = RC$)

\subsubsection{RC passa-alto}
\im{0.35}{"ims/CR1"}
La risposta in frequenza di questo circuito è:
$$H(f) = \frac{j2\pi f RC}{1 + j2\pi f RC}$$
da cui si nota che per frequenze basse $H_{db}\rightarrow-\infty$ e $\angle H\rightarrow\frac{\pi}{2}$, mentre per frequenze alte $H_{db}\rightarrow0$ e $\angle H \rightarrow0$.\\\\
Il comportamento del circuito è quindi il contrario del precedente.\\\\
I diagrammi di bode del circuito sono:
\im{0.8}{"ims/CR2"}
($\tau = RC$)



\newpage
\Index
\end{document}
